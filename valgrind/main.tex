\documentclass{beamer}
% \documentclass[draft]{beamer}

% Vista para impresión
% \usepackage{pgfpages}
% \pgfpagesuselayout{8 on 1}[a4paper,border shrink=5mm]

\mode<presentation>
{
  \usetheme{Frankfurt}
  \useinnertheme{rounded}
  \setbeamerfont{title}{shape=\itshape,family=\rmfamily}
  \setbeamercovered{transparent}
}

\usepackage[english]{babel}
\usepackage[latin1]{inputenc}

\usepackage{times}
\usepackage[T1]{fontenc}

\institute[IWay Tech Talks]{IWay Tech Talks}

\date[] {\today}

\title[Valgrind]{Valgrind: Profiling y debugging}
\subtitle{}
\author[]{Nicol\'as Brailovsky}

% This is only inserted into the PDF information catalog. Can be left out. 
\subject{Valgrind: Profiling y debugging para C/C++}


\pgfdeclareimage[height=1cm]{valgrind_logo}{valgrind_logo}
\pgfdeclareimage[height=4cm]{valgrind_logo_big}{valgrind_logo_big}
\logo{\pgfuseimage{valgrind_logo}}

% Delete this, if you do not want the table of contents to pop up at
% the beginning of each subsection:
\AtBeginSubsection[]
{
  \begin{frame}<beamer>{Outline}
    \tableofcontents[currentsection,currentsubsection]
  \end{frame}
}

\begin{document}

\begin{frame}[plain]
	\begin{center} \pgfuseimage{valgrind_logo_big} \end{center}
	\rule{0em}{5pt}

	\setbeamercolor{title}{fg=white,bg=teal}
	\begin{beamercolorbox}[sep=.1cm,shadow=true,rounded=true,center]{title}
		\inserttitle
	\end{beamercolorbox}

	\begin{columns}[onlytextwidth]
	\column{.5\textwidth}
		\begin{center}
			\pgfdeclareimage[height=1cm]{intraway}{intraway}
			\pgfuseimage{intraway}
		\end{center}
	\column{.5\textwidth}
		\begin{center} {\color{orange} \insertinstitute }  \end{center}
	\end{columns}

	\begin{center}
		{\color{orange} 
			\insertauthor \\
			\rule{0em}{8pt}
			\insertdate
		}
	\end{center}
\end{frame}

\begin{frame}
  \tableofcontents
\end{frame}


\section{Introducci\'on}

\begin{frame}{\textquestiondown Qu\'e es?}
\begin{itemize}
  \item Paquete de herramientas para analizar un programa en ejecuci\'on
  \begin{itemize}
    \item Memoria
    \item Performance
    \item File descriptors
    \item Race conditions
    \item \ldots
  \end{itemize}
  \item Un framework para desarrollar herramientas de an\'alisis
\end{itemize}
\end{frame}

\begin{frame}{\textquestiondown Para qu\'e utilizarlo?}
\begin{itemize}
  \item Complemento a herramientas de an\'alisis est\'atico
  \item Complemento para el gdb
  \item Detectar problemas "aleatorios"
  \item Detectar problemas que requieren largo tiempo de ejecuci\'on
\end{itemize}
\begin{center}
  \pgfdeclareimage[height=4cm]{debiancat3}{debiancat3}
  \pgfuseimage{debiancat3}
\end{center}
\end{frame}

\begin{frame}[t]
        \frametitle{\textquestiondown C\'omo funciona?}
\begin{itemize}
  \item Se crean wrappers para los syscall y para malloc/free 
  \item Cada bloque del heap lleva guardas
  \item Simula una CPU: analiza cada instrucci\'on
  \item Cada herramienta puede agregar distintas caracteristicas
\end{itemize}
\only<2> {
  \begin{center}
    \pgfdeclareimage[height=4cm]{gandalf}{gandalf}
    \pgfuseimage<2>{gandalf}
  \end{center}
}
\only<3>{
  \textit{ Gandalf dice: }
  \begin{center} \textit{
    Una aplicaci\'on bajo Valgrind consume ~ el \alert{triple de memoria} y
    es \alert{25 a 50 veces mas lento}
  } \end{center}
}
\end{frame}


\section{Uso general}

\begin{frame}{Flags generales}
\begin{itemize}
  \item \textit{ --tool=<toolname> } Herramienta a utilizar
  \item \textit{ --log-file=<archivo> } Redirigir la salida de Valgrind
  \item \textit{ --trace-children } Seguir un hijo despu\'es de fork/exec
  \item \textit{ --track-fds } Lista los FDs en uso 
  \item \textit{ --track-origins } Mostrar origen de valores no inicializados (V 3.4+)
  \item \textit{ -v } Verbose 
\end{itemize}
\end{frame}

\begin{frame}{Verificaciones comunes}
Independientemente de la herramienta, Valgrind \ldots
\begin{itemize}
  \item detecta una terminaci\'on erronea (segfaults!)
  \item advierte si existen memory leaks
  \item lleva cuenta de los file descriptors (con \textit{--track-fds})
  \item es lento
\end{itemize}
\end{frame}

\begin{frame}{Letra chica}
Independientemente de la herramienta, Valgrind \ldots
\begin{itemize}
  \item no se lleva bien con las variables est\'aticas
  \item tiene un limite de threads
  \item \alert{ es lento! }
\end{itemize}
\end{frame}


\section{Memcheck}

\begin{frame}{Memcheck: Introducci\'on}
\begin{itemize}
  \item Herramienta por default
  \item Busca memory leaks
  \item Detecta la utilizaci\'on de variables no inicializadas
  \item Advierte buffer overruns y problemas similares
\end{itemize}
\rule{0em}{20pt}
\pgfdeclareimage[height=2.5cm]{firefoxmemoryleak}{firefoxmemoryleak}
\pgfuseimage{firefoxmemoryleak}
\end{frame}

\begin{frame}[fragile, shrink=25]
\frametitle{Memcheck: Off by one}
\begin{semiverbatim}
\{
  char *msg = new char[4];
  strcpy(msg, "hola");
\}
\end{semiverbatim}

Error en valgrind:
\begin{semiverbatim}
\alert{ Invalid write } of size 1
   at 0x4024BD7: memcpy (mc_replace_strmem.c:402)
   by 0x80484DE: main \alert{ (main.cpp:5) }
 Address 0x42ae02c is \alert{ 0 bytes after a block of size 4 alloc'd }
   at 0x4022F14: operator new[](unsigned) (vg_replace_malloc.c:268)
   by 0x80484C0: main \alert{ (main.cpp:4) }
\end{semiverbatim}
\end{frame}

\begin{frame}[fragile, shrink=25]
\frametitle{Memcheck: Memory leak}
\textit{ Ejecutando el programa con --leack-check=full }
\begin{semiverbatim}
\{
  char *msg = new char[4];
  strcpy(msg, "hola");
\}
\end{semiverbatim}

Error en valgrind:
\begin{semiverbatim}
\alert{5 bytes in 1 blocks are definitely lost} in loss record 1 of 1
   at 0x4022F14: operator new[](unsigned) (vg_replace_malloc.c:268)
   by 0x80484C0: \alert{main (main.cpp:4)}

LEAK SUMMARY:
   \alert{ definitely lost: 5 bytes in 1 blocks. }
     possibly lost: 0 bytes in 0 blocks.
   still reachable: 0 bytes in 0 blocks.
        suppressed: 0 bytes in 0 blocks.
\end{semiverbatim}
\end{frame}

\begin{frame}[fragile, shrink=25]
\frametitle{Memcheck: Free/delete}
\begin{semiverbatim}
\{
  char *msg = new char[5];
  strcpy(msg, "hola");
  free(msg);
\}
\end{semiverbatim}

Error en valgrind:
\begin{semiverbatim}
\alert{Mismatched free() / delete / delete} []
   at 0x402265C: free (vg_replace_malloc.c:323)
   by 0x8048519: \alert{main (main.cpp:7)}
 Address 0x42ae028 is 0 bytes inside a block of size 5 alloc'd
   at 0x4022F14: operator new[](unsigned) (vg_replace_malloc.c:268)
   by 0x80484F0: main (main.cpp:5)
\end{semiverbatim}
\end{frame}

\begin{frame}[fragile, shrink=25]
\frametitle{Memcheck: Uninitialised value / SIGSEGV}
\begin{semiverbatim}
\{
  char *msg;
  strcpy(msg, "hola");
  free(msg);
\}
\end{semiverbatim}

Error en valgrind:
\begin{semiverbatim}
\alert{ Conditional jump or move depends on uninitialised value(s) }
   at 0x4024AC5: memcpy (mc_replace_strmem.c:77)
   by 0x804846F: main \alert{ (main.cpp:6) }

\alert{ Conditional jump or move depends on uninitialised value(s) }
   at 0x4024B7C: memcpy (mc_replace_strmem.c:80)
   by 0x804846F: main \alert{ (main.cpp:6) }
...
\alert{ Process terminating with default action of signal 11 (SIGSEGV) }
 Bad permissions for mapped region at address 0x400DDB0
   at 0x4024BA8: memcpy (mc_replace_strmem.c:402)
   by 0x804846F: \alert{ main (main.cpp:6) }
\end{semiverbatim}
\end{frame}

\begin{frame}[fragile, shrink=25]
\frametitle{Memcheck: Variable est\'atica}
\begin{semiverbatim}
char msg[4];
int main() \{
  strcpy(msg, "hola");
  return 0;
\}
\end{semiverbatim}

Error en valgrind:
\begin{center}
  \pgfdeclareimage[height=4cm]{404}{404}
  \pgfuseimage{404}
\end{center}
\end{frame}

\begin{frame}{Memcheck: conclusi\'on}
\begin{itemize}
  \item Una de las herramientas mas pr\'acticas
  \item Detecta el error y su origen
  \item Sencilla de utilizar
\end{itemize}
\textit{Para tener en cuenta}
\begin{itemize}
  \item Requiere compilar con -g (para ver la linea del error)
  \item No se lleva bien con -On: las optimizaciones generan falsos positivos
  \item No controla variables est\'aticas
\end{itemize}
\end{frame}



\section{Callgrind}

\begin{frame}{Callgrind: introducci\'on}
\begin{itemize}
  \item Extensi\'on de cachegrind
  \item Profiler para cache L1 y L2
  \item Crea un profile de cada funci\'on utilizada
  \item Permite generar gr\'aficos de llamada a partir de stacktraces
\end{itemize}
\rule{0em}{40pt}
\textit{ Para tener en cuenta \ldots }
\begin{itemize}
  \item es necesario utilizar KCacheGrind para interpretar los resultados
\end{itemize}
\end{frame}

\begin{frame}{Callgrind: ejemplo}
Gr\'afico de llamadas generado con KCacheGrind
\begin{center}
  \pgfdeclareimage[height=4cm]{callgraph}{callgraph}
  \pgfuseimage{callgraph}
\end{center}
\end{frame}

\begin{frame}[fragile]
\frametitle{Callgrind: ejemplo}
Utilizaci\'on del cache
\begin{verbatim}
--18822-- LRU Contxt Misses: 1672
--18822-- LRU BBCC Misses:   16
--18822-- LRU JCC Misses:    384
--18822-- BBs Executed:      717200001
--18822-- Calls:             281943034
--18822-- CondJMP followed:  0
--18822-- Boring JMPs:       0
--18822-- Recursive calls:   6
--18822-- Returns:           281943034
\end{verbatim}
\uncover<2>{ \begin{center} \alert{ RTFM } \end{center} }
\end{frame}

\begin{frame}{Callgrind: ejemplo}
Profiling de una aplicaci\'on (KCacheGrind)
\begin{center}
  \pgfdeclareimage[height=4cm]{call_usagegraph}{call_usagegraph}
  \pgfuseimage{call_usagegraph}
\end{center}
\rule{0em}{5pt}
\textquestiondown Qu\'e estar\'a haciendo zort()?
\end{frame}

\begin{frame}[fragile]
\frametitle{Callgrind: Ejemplo}
\begin{semiverbatim}
void {\color{green}zort}(vector<double> \&data) \{
  while (!{\color{green}sorted}(data))
    random_shuffle(data.begin(), data.end());
\}
\end{semiverbatim}
\begin{center} Parece que bogosort no es un buen algoritmo... \end{center}
\end{frame}

\begin{frame}{Callgrind: conclusi\'on}
\begin{itemize}
  \item Muestra que m\'etodos son hojas en el callgraph
  \item En una aplicaci\'on real callgrind encuentra bottlenecks
\end{itemize}
\end{frame}



\section{Massif}

\begin{frame}{Massif}
\begin{columns}[onlytextwidth]
\column{.5\textwidth}
  Massif sirve como \ldots
  \begin{itemize}
    \item Memory profiler
  \end{itemize}
\column{.5\textwidth}
  \pgfdeclareimage[height=3cm]{elephant}{elephant}
  \pgfuseimage{elephant}
\end{columns}
\end{frame}

\begin{frame}[fragile]
\frametitle{Massif: Ejemplo}
Implementemos este cambio para ver si Callgrind funciona bien
\begin{semiverbatim}
 \alert{ long } {\color{green}zort}(vector<double> \&data) \{
  \alert{ vector< vector<double> > intentos; }

  while (!{\color{green}sorted}(data)) \{
    std::random_shuffle(data.begin(), data.end());
    \alert{ intentos.push_back(vector<double>(data)); }
  \}

  \alert{ return intentos.size(); }
\}
\end{semiverbatim}
\end{frame}

\begin{frame}[fragile, shrink=45]
\frametitle{Massif: Ejemplo}
  Funciona, bogosort parece ser el culpable, pero \textquestiondown que pas\'o con la memoria?
\begin{verbatim}
    MB
106.0^                                      #                                 
     |                                      #                                 
     |                                     :#                          ,::::  
     |                                     :#                      ..::@::::  
     |                                    .:#                   . :::::@::::  
     |                                    ::#                .::: :::::@::::  
     |                                    ::#            .:: :::: :::::@::::  
     |                                   .::#        . ::::: :::: :::::@::::  
     |                                   :::#     , :: ::::: :::: :::::@::::  
     |                                   :::# ., :@ :: ::::: :::: :::::@::::  
     |                   @               :::# :@ :@ :: ::::: :::: :::::@::::: 
     |                   @            . ::::# :@ :@ :: ::::: :::: :::::@::::: 
     |                   @        ..::: ::::# :@ :@ :: ::::: :::: :::::@::::: 
     |                 . @     .: ::::: ::::# :@ :@ :: ::::: :::: :::::@::::: 
     |                 : @  : ::: ::::: ::::# :@ :@ :: ::::: :::: :::::@::::::
     |                .: @: : ::: ::::: ::::# :@ :@ :: ::::: :::: :::::@::::::
     |             .. :: @: : ::: ::::: ::::# :@ :@ :: ::::: :::: :::::@::::::
     |         :.:::: :: @: : ::: ::::: ::::# :@ :@ :: ::::: :::: :::::@::::::
     |    .. . :::::: :: @: : ::: ::::: ::::# :@ :@ :: ::::: :::: :::::@::::::
     |  ..:: : :::::: :: @: : ::: ::::: ::::# :@ :@ :: ::::: :::: :::::@::::::
   0 +----------------------------------------------------------------------->Gi
     0                                                                   4.714
\end{verbatim}
\end{frame}

\begin{frame}[fragile, shrink=45]
\frametitle{Massif: Ejemplo}
De los 106 MB, un 92\% corresponde a vector, reservado en \alert{massif.cpp:29}
\begin{verbatim}
92.45% (6,422,768B) (heap allocation functions) malloc/new/new[], --alloc-fns, etc.
->75.47% (5,243,120B) 0x8049561: __gnu_cxx::new_allocator<double>::allocate(unsigned, [...]
| ->75.47% (5,243,120B) 0x8049585: std::_Vector_base<double, std::allocator<double> [...]
[...]
|     
->16.98% (1,179,648B) 0x804963B: __gnu_cxx::new_allocator<std::vector<double, [...]
    [...]
        ->16.98% (1,179,648B) 0x8048C19: zort(std::vector<double, std::allocator<double> >&) (massif.cpp:18)
          ->16.98% (1,179,648B) 0x8048D48: main (massif.cpp:29)
\end{verbatim}
\end{frame}

\begin{frame}[fragile, shrink=45]
\frametitle{Massif: Un ejemplo mas completo}
\begin{verbatim}
    MB
2.033^                                    .      #      .      .      .      .
     |                                    :      #      :      :      :      :
     |                                    :      #      :      :      :      @
     |                                    :      #      :      :      :      @
     |                                    :      #      :      :      :      @
     |       @      ,      .      :      .:      #      :      :      :      @
     |       @      @      :      :      ::      #      :      :      :      @
     |       @      @      :      :      ::      #      :      :      :      @
     |       @      @      :      :      ::      #      :      :      :      @
     |       @      @      :      :      ::      #      :      :      :      @
     |@      @      @      :      :      ::      #      :      :      :      @
     |@      @      @      :      :      ::      #      :      :      :      @
     |@      @      @      :      :      ::      #      :      :      :      @
     |@      @      @      :      :      ::      #      :      :      :      @
     |@      @      @      :      :      ::      #      :      :      :      @
     |@      @      @      :      :      ::      #      :      :      :      @
     |@      @      @      :      :      ::      #      :      :      :      @
     |@      @      @      :      :      ::      #      :      :      :      @
     |@      @      @      :      :      ::      #      :      :      :      @
     |@      @      @      :      :      ::      #      :      :      :      @
   0 +----------------------------------------------------------------------->Gi
     0                                                                   7.850
\end{verbatim}
\end{frame}

\begin{frame}[fragile, shrink=45]
\frametitle{Massif: Un ejemplo mas completo}
Memoria utilizada
\begin{semiverbatim}
99.48\% (1,605,440B) (heap allocation functions) malloc/new/new, --alloc-fns, etc.
->65.48\% (1,056,768B) 0x80497B5: math::matrix<double>::base_mat::base_mat(unsigned, unsigned, double**) (in valgrind/mkv3)
| ->64.97\% (1,048,576B) 0x8049834: math::matrix<double>::matrix(unsigned, unsigned) (in valgrind/mkv3)
| | ->64.97\% (1,048,576B) 0x804A475: math::matrix<double>::operator*=(math::matrix<double> const\&) (in valgrind/mkv3)
| | | ->64.97\% (1,048,576B) 0x804A6CD: Mkv::get_tr_matrix() (in valgrind/mkv3)
| | |   ->32.49\% (524,288B) 0x804A864: Mkv::compare_to(Mkv\&) (in valgrind/mkv3)
| | |   | ->32.49\% (524,288B) 0x8049295: main (in valgrind/mkv3)
| | |   |   
| | |   \alert{ ->32.49\% (524,288B) 0x804A879: Mkv::compare_to(Mkv\&) (in valgrind/mkv3) }
| | |     ->32.49\% (524,288B) 0x8049295: main (in valgrind/mkv3)
| | |       
| | ->00.00\% (0B) in 1+ places, all below ms_print's threshold (01.00\%)
| | 
| ->00.51\% (8,192B) in 1+ places, all below ms_print's threshold (01.00\%)
|
->32.49\% (524,288B) 0x804960D: math::matrix<long>::base_mat::base_mat(unsigned, unsigned, long**) (in valgrind/mkv3)
| ->32.49\% (524,288B) 0x804968C: math::matrix<long>::matrix(unsigned, unsigned) (in valgrind/mkv3)
|  \alert{ ->32.49\% (524,288B) 0x8049AFE: Mkv::Mkv<unsigned char>(Tokenizer<unsigned char>\&) (in valgrind/mkv3) }
|     ->16.24\% (262,144B) 0x8049239: main (in valgrind/mkv3)
|     | 
|     ->16.24\% (262,144B) 0x804927D: main (in valgrind/mkv3)
|       
->01.02\% (16,384B) 0x40937EB: std::basic_filebuf<char, std::char_traits<char> >::_M_allocate_internal_buffer() (in /usr/lib/libstdc++.so.6.0.9)
| ->01.02\% (16,384B) 0x4097231: std::basic_filebuf<char, std::char_traits<char> >::open(char const*, std::_Ios_Openmode) (in /usr/lib/libstdc++.so.6.0.9)
|   ->01.02\% (16,384B) 0x4097C0E: std::basic_ifstream<char, std::char_traits<char> >::basic_ifstream(char const*, std::_Ios_Openmode) (in /usr/lib/libstdc++.so.6.0.9)
|     ->01.02\% (16,384B) 0x8049499: FileTokenizer<unsigned char>::FileTokenizer(char const*) (in valgrind/mkv3)
|       \alert{ ->01.02\% (16,384B) 0x80494B3: CharTokenizer::CharTokenizer(char const*) (in valgrind/mkv3) }
|         ->01.02\% (16,384B) in 2 places, all below massif's threshold (01.00\%)
|           
->00.50\% (8,000B) in 1+ places, all below ms_print's threshold (01.00\%)

\end{semiverbatim}
\end{frame}


\begin{frame}{Massif: conclusi\'on}
\begin{itemize}
  \item Herramienta complementaria a Callgrind
  \item Permite detectar bottlenecks
  \item Los cambios en la \'ultima versi\'on no generan postscript
  \item El formato actual es horrible!
\end{itemize}
\end{frame}



\section{Helgrind}

\begin{frame}{Helgrind: Introducci\'on}
\begin{columns}[onlytextwidth]
\column{.5\textwidth}
  \pgfdeclareimage[height=4cm]{deadlock}{deadlock}
  \pgfuseimage{deadlock}
\column{.5\textwidth}
  Helgrind muestra \ldots
  \begin{itemize}
    \item condiciones de carrera
    \item deadklocks
  \end{itemize}
\end{columns}
\end{frame}

\begin{frame}[fragile]
\frametitle{Helgrind: Ejemplo}
Programa de ejemplo:
\begin{semiverbatim}

class {\color{green}Hormiguita} : public {\color{green}ACE_Task_Base} \{
  public:
  int \&num;
  {\color{green}Hormiguita}(int \&num) : num(num) \{
    activate(..., 10);
  \}

  int svc() \{
    \alert{ while (num < 1000) num++; }
    return 0;
  \}
\};
\end{semiverbatim}
\end{frame}

\begin{frame}[fragile,shrink=35]
\frametitle{Helgrind: Ejemplo}
\small{
\begin{semiverbatim}
==2404== Possible \alert{data race during read} of size 4 at 0xbee16e78 by thread #3
==2404==    at 0x8049D4D: Worker::svc() \alert{(helgrind.cpp:20)}
==2404==    by 0x413D551: ACE_Task_Base::svc_run(void*) (Task.cpp:275)
==2404==    by 0x413E287: ACE_Thread_Adapter::invoke_i() (Thread_Adapter.cpp:149)
==2404==    by 0x413E465: ACE_Thread_Adapter::invoke() (Thread_Adapter.cpp:98)
==2404==    by 0x40CC2D0: ace_thread_adapter (Base_Thread_Adapter.cpp:124)
==2404==    by 0x4026BFE: mythread_wrapper (hg_intercepts.c:194)
==2404==    by 0x441F4FA: start_thread (in /lib/tls/i686/cmov/libpthread-2.7.so)
==2404==    by 0x43A1E5D: clone (in /lib/tls/i686/cmov/libc-2.7.so)
==2404==  This \alert{conflicts with a previous write} of size 4 by thread #2
==2404==    at 0x8049D44: Worker::svc() \alert{(helgrind.cpp:20)}
==2404==    by 0x413D551: ACE_Task_Base::svc_run(void*) (Task.cpp:275)
==2404==    by 0x413E287: ACE_Thread_Adapter::invoke_i() (Thread_Adapter.cpp:149)
==2404==    by 0x413E465: ACE_Thread_Adapter::invoke() (Thread_Adapter.cpp:98)
==2404==    by 0x40CC2D0: ace_thread_adapter (Base_Thread_Adapter.cpp:124)
==2404==    by 0x4026BFE: mythread_wrapper (hg_intercepts.c:194)
==2404==    by 0x441F4FA: start_thread (in /lib/tls/i686/cmov/libpthread-2.7.so)
==2404==    by 0x43A1E5D: clone (in /lib/tls/i686/cmov/libc-2.7.so)
\end{semiverbatim}
}
\end{frame}

\begin{frame}{Helgrind: conclusi\'on}
\begin{itemize}
  \item Una excelente idea, pero \ldots
  \item el S/N ratio es muy bajo, mas a\'un en ACE \footnote{ \\
        En el slide anterior solo se muestra una parte del log}
\end{itemize}
\end{frame}



% All of the following is optional and typically not needed. 
\appendix
\section<presentation>*{\appendixname}
\subsection<presentation>*{Further reading}

\begin{frame}[allowframebreaks]
  \frametitle<presentation>{Further reading}
    
  \begin{thebibliography}{10}
    
  \beamertemplatebookbibitems
  \bibitem{RTFM}
    RTFM
    \newblock {\em man valgrind }

    
  \beamertemplatearticlebibitems

  \beamertemplatearticlebibitems
  \bibitem{Valgrind site}
    2000-2009 Valgrind Developers
    \newblock Valgrind website
    \newblock \htmladdnormallink{http://valgrind.org}{http://valgrind.org}

  \end{thebibliography}
\end{frame}

\end{document}
