\section{Introducci\'on}

%%%%%%%%%%%%%%%%%%%%%%%%%%%%%%%%%%%%%%%%%%%%%%%%%%

\subsection{Planteando el problema}


\begin{frame}{Cuando escribimos software}
\begin{columns}[onlytextwidth]
\column{.5\textwidth}
	\pgfdeclareimage[height=5cm]{img/misc/escher-relativity}{img/misc/escher-relativity}
	\pgfuseimage{img/misc/escher-relativity}
\column{.5\textwidth}
	\begin{itemize}
		\item falta tiempo
		\item hacemos suposiciones
		\item el c\'odigo queda feo
		\item proyectos legacy sin andar
		\item falta tiempo de refactor
	\end{itemize}
\end{columns}
\begin{center}
	Por todo esto la calidad se degrada
\end{center}
\end{frame}

\begin{frame}{Si bien no hay silver bullets}
Veremos como, testeando, el proyecto sale mejor. Los tests:
\bigskip
\begin{columns}[onlytextwidth]
\column{.15\textwidth}
	\pgfdeclareimage[height=3cm]{img/misc/silver-bullet}{img/misc/silver-bullet}
	\pgfuseimage{img/misc/silver-bullet}
\column{.85\textwidth}
	\begin{itemize}
		\item generan confianza
		\item permiten la detecci\'on temprana de bugs
		\item ayudan a detectar inconsistencias en RQs
		\item mejoran la productividad (a largo plazo)
	\end{itemize}
\end{columns}
\end{frame}

\begin{frame}{Si bien no hay silver bullets}
Veremos como, testeando, el proyecto sale mejor. \\
\alert{ Siempre considerando que, los tests \ldots }
\bigskip
\begin{columns}[onlytextwidth]
\column{.15\textwidth}
	\pgfdeclareimage[height=3cm]{img/misc/silver-bullet}{img/misc/silver-bullet}
	\pgfuseimage{img/misc/silver-bullet}
\column{.85\textwidth}
	\begin{itemize}
		\item no garantizan el \'exito de un proyecto
		\item no aseguran 100\% de calidad
		\item no son aplicables a todo proyecto
		\item obligan a pensar antes de programar! \footnote{Queda como ejercicio determinar si esto es bueno o no}
	\end{itemize}
\end{columns}
\bigskip
Pero, \textquestiondown Qu\'e es testing?
\end{frame}

%%%%%%%%%%%%%%%%%%%%%%%%%%%%%%%%%%%%%%%%%%%%%%%%%%

\subsection{\textquestiondown Qu\'e es testing?}

\begin{frame}{\textquestiondown Qu\'e es testing?}
\begin{columns}[onlytextwidth]
\column{.7\textwidth}
	Testear el software es:
	\begin{itemize}
		\item validar su comportamiento
		\item validar un set de datos
		\item validar requerimientos
		\item una forma de medir la calidad
	\end{itemize}
	\bigskip
	Testear el software \alert{no} es:
	\begin{itemize}
		\item asegurar el \'exito del proyecto
		\item afirmar que no hay bugs
		\item "a silver bullet"
	\end{itemize}
\column{.3\textwidth}
	\pgfdeclareimage[height=4cm]{img/misc/crash-dummy}{img/misc/crash-dummy}
	\pgfuseimage{img/misc/crash-dummy}
\end{columns}
\end{frame}


\begin{frame}[fragile,shrink=2]{Algunos "extras"}
\bigskip
El testing (o, mejor, TDD) ayuda a:
\begin{columns}[onlytextwidth]
\column{.5\textwidth}
	\begin{itemize}
		\item especificar requerimientos
		\item especificar comportamiento
		\item evitar regresiones
		\item refactorizar
	\end{itemize}
\column{.5\textwidth}
	\begin{flushright}
		\pgfdeclareimage[height=3.5cm]{img/misc/test-drive}{img/misc/test-drive}
		\pgfuseimage{img/misc/test-drive}
	\end{flushright}
\end{columns}

\begin{textblock*}{30mm}(-15mm,-2mm)
	\pgfdeclareimage[height=3.5cm]{img/misc/crash-dummies}{img/misc/crash-dummies}
	\pgfuseimage{img/misc/crash-dummies}
\end{textblock*}
\end{frame}

\begin{frame}{C\'omo testear}
\begin{center}
	Vimos muchos motivos, muy dispares \ldots \\ 
	\textquestiondown todo eso con un test?
\end{center}

\begin{center}
	No, hay distintos tipos. Veamos algunos.
\end{center}
\end{frame}

%%%%%%%%%%%%%%%%%%%%%%%%%%%%%%%%%%%%%%%%%%%%%%%%%%


