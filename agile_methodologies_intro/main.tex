\documentclass{beamer}
% \documentclass[draft]{beamer}

% Vista para impresión
% \usepackage{pgfpages}
% \pgfpagesuselayout{8 on 1}[a4paper,border shrink=5mm]

\mode<presentation>
{
  \usetheme{Frankfurt}
  \useinnertheme{rounded}
  \setbeamerfont{title}{shape=\itshape,family=\rmfamily}
  \setbeamercovered{transparent}
}

\usepackage[english]{babel}
\usepackage[latin1]{inputenc}

\usepackage{times}
\usepackage[T1]{fontenc}

\institute[Universidad Tecnol\'ogica Nacional]
{Universidad Tecnol\'ogica Nacional \\ Facultad Regional Buenos Aires \\ Dise\~no de Sistemas}

\date[] {\today}

\title[Metodolog\'ias \'Agiles]{Introducci\'on a las Metodolog\'ias \'Agiles}
\subtitle{}
\author[]
{
  Nicol\'as Brailovsky
%  \and 
%  ???
}

% This is only inserted into the PDF information catalog. Can be left out. 
\subject{Metodolog\'ias \'Agiles}

\pgfdeclareimage[height=0.5cm]{university-logo}{logo-utn}
\pgfdeclareimage[height=1.2cm]{university-logo-big}{logo-utn}
\logo{\pgfuseimage{university-logo}}


% Delete this, if you do not want the table of contents to pop up at
% the beginning of each subsection:
\AtBeginSubsection[]
{
  \begin{frame}<beamer>{Outline}
    \tableofcontents[currentsection,currentsubsection]
  \end{frame}
}

\begin{document}

\begin{frame}[plain]
  \begin{columns}[onlytextwidth]
  \column{.8\textwidth}
      \insertinstitute
  \column{.2\textwidth}
    \begin{flushright} \pgfuseimage{university-logo-big} \end{flushright}
  \end{columns}
  \rule{0em}{25pt}

  \setbeamercolor{title}{fg=white,bg=teal}
  \begin{beamercolorbox}[sep=.1cm,shadow=true,rounded=true,center]{title}
    \inserttitle
  \end{beamercolorbox}
  \par
  \rule{0em}{20pt}

  \begin{center}
    \insertauthor \\
    \rule{0em}{8pt}
    \insertdate
  \end{center}
\end{frame}

\begin{frame}
  \tableofcontents
\end{frame}


\section{\textquestiondown Qu\'e es una metodolog\'ia?}

\begin{frame}{}
\textquestiondown Qu\'e es una metodolog\'ia \ldots
\begin{itemize}
  \item en general?
  \item aplicada al desarrollo de software?
  \item \'agil?
\end{itemize}
\end{frame}

\begin{frame}{}
\pgfdeclareimage[height=4cm]{traditional-methodology}{traditional-methodology}
\begin{center}
\pgfuseimage{traditional-methodology}
\end{center}
\end{frame}

\begin{frame}{\textquestiondown Para qu\'e?}
\textit{En las metodolog\'ias "pesadas" el desarrollo de software se ve}
\begin{itemize}
  \item como un proceso de manufactura
  \item predecible y estable
  \item estructurado en forma r\'igida
\end{itemize}
\textit{Resultados}
\begin{itemize}
  \item Productos de poca calidad
  \item Proyectos inestables
  \item Problemas de motivaci\'on en el equipo
\end{itemize}
\end{frame}

\begin{frame}{\textquestiondown Qu\'e es una metodolog\'ia \'agil?}
\pgfdeclareimage[height=3.5cm]{dilbert-agile}{dilbert-agile}
\begin{center}
\pgfuseimage{dilbert-agile}
\end{center}
\begin{flushright}\begin{color}{green}
  // TODO: Pedirle a Scott que nos deje usar esta imagen
\end{color}\end{flushright}
\end{frame}



\section{Metodolog\'ias \'Agiles}

\begin{frame}{Manifesto \'Agil}
\begin{itemize}
  \item Individuos e interacciones $>$ procesos y herramientas
  \item Software en funcionamiento $>$ documentaci\'on comprehensiva
  \item Colaboraci\'on del cliente $>$ la negociaci\'on de un contrato
  \item Responder al cambio $>$ seguir un plan
\end{itemize}
\begin{flushright}
  \htmladdnormallink{agilemanifesto.org}{http://agilemanifesto.org}
\end{flushright}
\end{frame}

\begin{frame}{Manifesto \'Agil}
\textit{El "software en funcionamiento" es solo parte de los entregables}
\begin{itemize}
  \item \textquestiondown Qu\'e otros entregables existen?
  \item Un manual de usuario, o la documentaci\'on de una API, \\
        \begin{itemize}
                \item \textquestiondown son entregables?
                \item \textquestiondown son software en funcionamiento?
        \end{itemize}
\end{itemize}
\textit{Se puede clasificar la documentaci\'on como hist\'orica y para soporte del proceso}
\begin{itemize}
  \item \textquestiondown A cu\'al se refiere el manifesto?
  \item \textquestiondown Sin documentaci\'on comprehensiva == No documentar?
\end{itemize}
\end{frame}

\begin{frame}{Distintos enfoques}
\begin{itemize}
  \item SCRUM
  \item XP, Extreme Programming
  \item Lean Software Development
  \item Crystal
\end{itemize}
\end{frame}



\section{Scrum}

\begin{frame}{Introducci\'on}
\begin{itemize}
  \item Metodolog\'ia iterativa y adaptativa al proyecto
  \item Centrado en la gesti\'on del proyecto
  \item Gestiona requerimientos y tiempos de entrega
\end{itemize}
\end{frame}

\begin{frame}{Gu\'ia del proyecto}
\pgfdeclareimage[height=6cm]{scrum-project}{scrum-project}
\begin{center}
\pgfuseimage{scrum-project}
\end{center}
\end{frame}

\begin{frame}{Sprint}
\textit{Un proceso guiado por Scrum se basa en una serie de sprints}
\begin{itemize}
        \item Ciclo corto (2 o 3 semanas)
        \item Involucra a stakeholders, el equipo de desarrollo y un Scrum master
        \item Los stakeholders, con el grupo, crean y priorizan una lista de requerimientos
        \item La lista priorizada de requerimientos para el proyecto se llama product backlog
        \item La misma lista para el sprint es el sprint backlog
        \item El equipo se compromete a cumplir una determinada cantidad de RQs por sprint
\end{itemize}
\end{frame}

\begin{frame}{Sprint}
\begin{itemize}
        \item La cantidad de historias por sprint determina la velocidad del equipo
        \item La velocidad del equipo sirve para estimar el siguiente sprint backlog
        \item La velocidad se basa en datos hist\'oricos: no es una estimaci\'on
        \item La velocidad no se mide, necesariamente, en unidades de tiempo
        \item Todos cometemos errores al estimar pero, estad\'isticamente, el error es consistente
        \item La estimaci\'on se puede realizar con un juego de estimaci\'on
        \item Al final del sprint debe existir un entregable con valor para los stakeholders
\end{itemize}
\end{frame}



\section{XP}

\begin{frame}{Introducci\'on}
\begin{columns}
\column{.5\textwidth}
  \begin{itemize}
    \item Mejores pr\'acticas
    \item Orientado al d\'ia a d\'ia
    \item \'Enfasis en el testing
    \item Nombre marketinero (?)
  \end{itemize}
\column{.5\textwidth}
  \pgfdeclareimage[height=4cm]{xp-intro}{xp-intro}
  \begin{center}\pgfuseimage{xp-intro}\end{center}
\end{columns}
\end{frame}

\begin{frame}{Gu\'ia de proyecto}
\pgfdeclareimage[height=4cm]{xp-project}{xp-project}
\begin{center}
\pgfuseimage{xp-project}
\end{center}
\end{frame}

\begin{frame}{Principios}
\textit{XP basa sus pr\'acticas en}
\begin{itemize}
        \item Comunicaci\'on, Feedback, Simplicidad, Valor, Respeto
\end{itemize}
\textit{Se adapta a proyectos con}
\begin{itemize}
        \item requerimientos inestables
        \item alto riesgo
        \item equipos con poca experiencia
        \item equipos chicos
\end{itemize}
Una de las inovaciones de XP es integrar al testing como parte del proceso y no como tarea de soporte
\end{frame}

\begin{frame}{Pr\'acticas}
\begin{itemize}
        \item Las historias de usuario sirven para crear test de aceptaci\'on y para estimar
        \item Entregables en ciclos cortos en incrementos peque\~nos
        \item Propiedad colectiva del c\'odigo / Rotaci\'on interna del equipo
        \item Pair programming
        \item El cliente siempre est\'a disponible
        \item Testing antes de codificar (TDD/BDD)
        \item Integraci\'on continua
        \item Refactor and consolidate
        \item KISS
\end{itemize}
\end{frame}

\begin{frame}{Pr\'acticas}
\pgfdeclareimage[height=6cm]{xp-practices}{xp-practices}
\begin{center}
\pgfuseimage{xp-practices}
\end{center}
\end{frame}
  

\subsection{Test Driven Development}

\begin{frame}{Introducc\'ion}
TDD: Una de las pr\'acticas mas inovadoras de XP
\textquestiondown Para qu\'e escribir tests?
\begin{itemize}
          \item Reutilizaci\'on de c\'odigo $\Rightarrow$ bueno
          \item Tocar en un lado y que explote todo $\Rightarrow$ malo
\end{itemize}
Los tests son una forma de asegurar que la interfaz del m\'odulo no cambia.
Adem\'as:
\begin{itemize}
        \item Es una forma de documentar las historias de usuario
        \item No solo documenta, sirve como ejemplo!
        \item Incrementa la confianza en la fiabilidad del sistema
        \item Obliga al programador a pensar antes de codificar
        \item El trabajo termina una vez que todos los tests pasan
\end{itemize}
\end{frame}

\begin{frame}{Metodolog\'ia de trabajo}
La metodolog\'ia de trabajo en TDD se conoce como "Red - Green - Refactor"
\begin{itemize}
        \item Escribir tests como especificaci\'on de un comportamiento
        \item Correr los tests; ver como fallan
        \item Implementar c\'odigo
        \item Cuando los tests pasan (est\'an en verde) el desarrollo est\'a completo
        \item Ahora se puede refactorizar el c\'odigo, sin miedo a romper nada
\end{itemize}
\end{frame}

\begin{frame}{Behaviour Driven Development}
Utilizando TDD
\begin{itemize}
        \item se puede detectar una falla con el mismo cambio que la provoca
        \item se pretende definir comportamiento, no solo verificar la estructura del c\'odigo
\end{itemize}
Esto se conoce como "BDD", Behaviour Driven Development: definir el comportamiento del sistema a partir de un test.
\begin{itemize}
        \item Con TDD/BDD los tests deben correr en un tiempo razonable, es decir, se deben utilizar tests unitarios (no de integraci\'on!)
\end{itemize}
\end{frame}

\subsection{Integraci\'on Continua}

\begin{frame}[t]
  \frametitle{Introducci\'on}
\begin{center}
  En sistemas grandes el esfuerzo se divide en componentes, equipos, etc.
  \textquestiondown Qu\'e pasa al intentar integrar todas las partes?
  \pgfdeclareimage[height=4cm]{nuclear}{nuclear}
  \pgfuseimage<2>{nuclear}
\end{center}
\end{frame}

\begin{frame}{Pr\'acticas}
\begin{itemize}
        \item La \'ultima versi\'on del repositorio SIEMPRE compila \footnote{ Pr\'actica de XP: Quien rompa el build deber\'a traer facturas }
        \item Los test de integraci\'on no deben tardar mas de 5 minutos
        \item Permite detectar errores de integraci\'on minutos (~5) despu\'es del commit que lo caus\'o
\end{itemize}

\end{frame}

% All of the following is optional and typically not needed. 
\appendix
\section<presentation>*{\appendixname}
\subsection<presentation>*{Lectura adicional}

\begin{frame}[allowframebreaks]
  \frametitle<presentation>{Lectura Adicional}
    
  \begin{thebibliography}{10}
    
  \beamertemplatebookbibitems
  \bibitem{PRESS}
    Pressman, Roger
    \newblock {\em Ingenier\'ia del Software: Un enfoque Pr\'actico}.
    \newblock Mc Graw Hill.

    
  \bibitem{DUVALL07}
    Duvall, Paul et al.
    \newblock {\em Continuous Integration}.
    \newblock Addison Wesly, 2007

  \beamertemplatearticlebibitems
%  \beamertemplatearticlebibitems
%  \bibitem{Someone2000}
%    S.~Someone.
%    \newblock On this and that.
%    \newblock {\em Journal of This and That}, 2(1):50--100,
%    2000.

  \bibitem{agilemanifesto}
    Agile Manifesto
    \newblock \htmladdnormallink{http://agilemanifesto.org}{http://agilemanifesto.org}

  \beamertemplatearticlebibitems
  \bibitem{XP-Homepage}
    Extreme Programming: A Gentle Introduction
    \newblock \htmladdnormallink{http://www.extremeprogramming.org/}{http://www.extremeprogramming.org/}

  \beamertemplatearticlebibitems
  \bibitem{FOW1}
    Martin Fowler
    \newblock The New Methodology
    \newblock \htmladdnormallink{http://www.martinfowler.com/articles/newMethodology.html}{http://www.martinfowler.com/articles/newMethodology.html}

  \beamertemplatearticlebibitems
  \bibitem{FOW2}
    Martin Fowler
    \newblock Continuous Integration
    \newblock \htmladdnormallink{http://www.martinfowler.com/articles/continuousIntegration.html}{http://www.martinfowler.com/articles/continuousIntegration.html}

  \beamertemplatearticlebibitems
  \bibitem{FOW3}
    Martin Fowler
    \newblock Is Design Dead?
    \newblock \htmladdnormallink{http://www.martinfowler.com/articles/designDead.html}{http://www.martinfowler.com/articles/designDead.html}
    
  \beamertemplatearticlebibitems
  \bibitem{BECK0}
    Kent Beck
    \newblock Extreme Programming
    \newblock \htmladdnormallink{http://c2.com/cgi/wiki?ExtremeProgramming}{http://c2.com/cgi/wiki?ExtremeProgramming}
    
  \beamertemplatearticlebibitems
  \bibitem{SHORE0}
    James Shore
    \newblock The Art of Agile
    \newblock \htmladdnormallink{http://jamesshore.com/}{http://jamesshore.com/}
    
  \end{thebibliography}
\end{frame}

\end{document}
