\documentclass{beamer}
% \documentclass[draft]{beamer}

% Vista para impresión
% \usepackage{pgfpages}
% \pgfpagesuselayout{8 on 1}[a4paper,border shrink=5mm]

\mode<presentation>
{
  \usetheme{Frankfurt}
  \useinnertheme{rounded}
  \setbeamerfont{title}{shape=\itshape,family=\rmfamily}
  \setbeamercovered{transparent}
}

\usepackage[english]{babel}
\usepackage[latin1]{inputenc}

\usepackage{times}
\usepackage[T1]{fontenc}

\institute[IWay Tech Talks]{IWay Tech Talks}

\date[] {\today}

\title[Tips para una presentaci\'on copada]{(Meta?) Iway Tech Talk: Tips para una presentaci\'on copada}
\subtitle{}
\author[]{Nicol\'as Brailovsky}

% This is only inserted into the PDF information catalog. Can be left out. 
\subject{}


\pgfdeclareimage[height=1cm]{pinocho}{pinocho}
\pgfdeclareimage[height=4cm]{pinocho_big}{pinocho_big}
\logo{\pgfuseimage{pinocho}}

% Delete this, if you do not want the table of contents to pop up at
% the beginning of each subsection:
\AtBeginSubsection[]
{
  \begin{frame}<beamer>{Outline}
    \tableofcontents[currentsection,currentsubsection]
  \end{frame}
}

\begin{document}

\begin{frame}[plain]
	\begin{center} \pgfuseimage{pinocho_big} \end{center}
	\rule{0em}{5pt}

	\setbeamercolor{title}{fg=white,bg=teal}
	\begin{beamercolorbox}[sep=.1cm,shadow=true,rounded=true,center]{title}
		\inserttitle
	\end{beamercolorbox}

	\begin{columns}[onlytextwidth]
	\column{.5\textwidth}
		\begin{center}
			\pgfdeclareimage[height=1cm]{intraway}{intraway}
			\pgfuseimage{intraway}
		\end{center}
	\column{.5\textwidth}
		\begin{center} {\color{orange} \insertinstitute }  \end{center}
	\end{columns}

	\begin{center}
		{\color{orange} 
			\insertauthor \\
			\rule{0em}{8pt}
			\insertdate
		}
	\end{center}
\end{frame}

\begin{frame}
  \tableofcontents
\textit{Perfection is achieved, not when there is nothing left to add, \
         but when there is nothing left to remove. }
\begin{center}\textit{ -- Antoine de Saint-Exupery} \end{center}
\end{frame}


\section{Introducci\'on}

% En internet hay un monton de informacion sobre como dar presentaciones, 
% con muchos tips copados. Ademas hay un monton de gente que seguro sabe
% mucho mas que yo del tema, con mucha mas practica dando presentaciones.
% Que hacemos aca entonces? Mas importante, que hago yo aca adelante?
% Primero, seguramente estemos aca porque soy un caradura. Pero tambien
% estamos aca porque llevo un tiempo dando presentaciones y fui descubriendo
% algunos tips que por ahi a alguien mas le pueden servir. Y mas importante
% todavia, estamos aca porque seguramente muchos de ustedes tengan sus
% propios tips, y estaria bueno que los compartan tambien\ldots
\begin{frame}{Qu\'e tips?}
\begin{columns}[onlytextwidth]
\column{.35\textwidth}
    \begin{center}
        \pgfdeclareimage[height=4.5cm]{tron}{tron}
        \pgfuseimage{tron}
    \end{center}
\column{.65\textwidth}
    \begin{itemize}
        \item La Internet tiene mucha info \ \bigskip
        \item La Internet es aburrida \ \bigskip
        \item y no queremos reinventar la rueda \ \bigskip
    \end{itemize}
\end{columns}
\end{frame}


% Empecemos por decir que es lo que no vamos a ver. Hay demasiadas cosas
% copiables y tips recontraconocidos. La idea es no ver ninguno de esos sino
% pasar directamente a cosas un poco mas jugosas. Decia que no soy un experto
% en el tema, pero despues de algunos a\~nos de dar presentaciones fui
% recopilando algunos tips y guidelines interesantes para armar una presentacion.
% La idea es compartir esos tips y, si se animan, que la audiencia comparta los
% suyos.
\begin{frame}{Qu\'e NO vamos a ver?}
\begin{columns}[onlytextwidth]
\column{.65\textwidth}
    \begin{itemize}
        \item Powerpoint templates \ \bigskip
        \item LaTeX (esa charla la da Adri\'an) \ \bigskip
        \item C\'omo copiar a Steve Jobs \ \bigskip
        \item Todos los tips est\'andar\ldots
    \end{itemize}
\column{.35\textwidth}
    \begin{center}
        \pgfdeclareimage[height=4.5cm]{guitarra}{guitarra}
        \pgfuseimage{guitarra}
    \end{center}
\end{columns}
\end{frame}


% Una parte bastante importante de las charlas suele ser la audiencia.
% Hablar sin que nadie nos escuche puede ser considerado por algunos
% como una conversaci\'on con alguien inteligente pero para otros es motivo
% de traslado al borda. Qu\'e deber\'iamos considerar sobre nuestra
% audiencia para una presentaci´on copada?
\section{Sobre la audiencia}

% Conocer la audiencia es importante. Principalmente para no hacer (mucho)
% papel\'on. Por ejemplo, no estar\'ia bueno hablar de reposter\'ia en un curso de
% CakePHP? (Ehh... solo a mi me paso?). Un punto importante sobre la audiencia
% es la capacidad de atenci\'on. En general el tiempo efectivo de atenci\'on
% no pasa de los 45 minutos, despues de eso la audiencia comienza a experimentar
% alucinaciones por privacion sensorial.
\begin{frame}{Tiempo de atenci\'on}
\only<1> {
    \begin{itemize}
        \item Sea $A$ el tiempo de atenci\'on $ => A = K / t$ \ \bigskip
        \item el tiempo efectivo es de 45 minutos! \ \bigskip
        \item aunque para algunos es efectivamente cero\ldots
    \end{itemize}
} \only<2> {
    \begin{center}
        \pgfdeclareimage[height=7cm]{atencion_por_t}{atencion_por_t}
        \pgfuseimage{atencion_por_t}
    \end{center}
}
\end{frame}


% Que podemos hacer para mejorar un poco este tiempo de atencion? La opcion
% mas efectiva es eliminar todo tipo de distracciones. Atar a la audiencia a las
% sillas, apagar las luces, usar un cuarto insonorizado, ese tipo de cosas.
% Para los que disfrutan de los desafios, algo un poco menos efectivo pero mas
% practico es mejorar la presentacion. El primer tip es agregar un soporte visual
% interesante, algo para que la gente pueda distraer un poco la vista cuando se
% cansan de escucharnos.
\begin{frame}{Tiempo de atenci\'on}
\begin{columns}[onlytextwidth]
\column{.21\textwidth}
    \begin{center}
        \pgfdeclareimage[height=3.5cm]{butterfly}{butterfly}
        \pgfuseimage{butterfly}
    \end{center}
\column{.79\textwidth}
    \begin{itemize}
        \item Sea $A$ el tiempo de atenci\'on $ => A = f(|c|)$ \ \bigskip
        \item Siendo $c$ = \{ fotos rid\'iculas de gatos \} \ \bigskip
    \end{itemize}
\end{columns}
\end{frame}


% Que podemos hacer para maximir el tiempo de atencion, evitar aburrir a la
% audiencia y disminuir las chances de que suban a youtube un video nuestro?
% Una buena idea es el soporte visual. Las fotos de gatos son excelente.
\begin{frame}{Mejorando el tiempo de atenci\'on: un ejemplo}
\begin{center}
\only<1> {
    \pgfdeclareimage[height=5cm]{cat}{cat}
    \pgfuseimage{cat}
} \only<2> {
    Ehm\ldots me refer\'ia a este tipo, pero tambi\'en sirve: \
    \pgfdeclareimage[height=5cm]{realcat}{realcat}
    \pgfuseimage{realcat}
} \only<3> {
    \pgfdeclareimage[height=7cm]{atencion_por_t_ej}{atencion_por_t_ej}
    \pgfuseimage{atencion_por_t_ej}
}
\end{center}
\end{frame}


% Si bien el apoyo visual ayuda a mantener la atenci\'on, cuando el contenido
% de la charla es vacio la atencion se pierde. Si bien hay excelentes vendedores
% que son capaces, yo no puedo estar al frente de un auditorio contando el cuento
% de la buena pipa. Que podemos hacer entonces para mejorar nuestro contenido?
% (una vez que ya encontramos suficientes lolcats, claro)
\begin{frame}{Tiempo de atenci\'on}
\begin{center}
    \textit{ Claro\ldots los gatitos ayudan\ldots si el contenido acompa\~na! }
    \pgfdeclareimage[height=5cm]{invisiblecat}{invisiblecat}
    \pgfuseimage{invisiblecat}
\end{center}
\end{frame}


\section{Mejorando el contenido}


% Hay varias formas de presentar el contenido de la charla para
% mejorar la atenci\'on de la audiencia, pero en general se basan
% todas en los mismos principios: generar expectativa en el oyente,
% es decir, hacer que quien nos escucha se pregunte activamente
% que es lo siguiente que vamos a decir, o generar conflicto, es decir
% plantear una situacion que a la audiencia le resulta dificil de creer
% y luego demostrar porque esto es cierto.
\begin{frame}{Tips m\'agicos para el contenido}
\begin{columns}[onlytextwidth]
\column{.3\textwidth}
    \begin{center}
        \pgfdeclareimage[height=3.5cm]{gandalf}{gandalf}
        \pgfuseimage{gandalf}
    \end{center}
\column{.7\textwidth}
    \begin{itemize}
        \item Mejor contenido $=>$ m\'as atenci\'on \ \bigskip
        \item Todos los tips se basan en lo mismo \ \bigskip \
            \begin{itemize}
                \item Expectativa \ \bigskip
                \item Conflicto \ \bigskip
                \item Sorpresa
            \end{itemize}
    \end{itemize}
\end{columns}
\end{frame}


% El factor sorpresa es un elemento importante para mantener la atencion
% en las presentaciones y para peliculas de terror clase Z. Bien utilizado
% nos permite agregar alguna situacion comica y prolongar un poco el tiempo
% de atencion de la audiencia. Que hubiera pasado si el mismo chiste del gato
% lo hacia durante una presentacion en la asociacion nacional feminista? Mal
% utilizado, podemos ofender a mucha gente.
\begin{frame}{Suprize!}
\begin{columns}[onlytextwidth]
\column{.7\textwidth}
    \begin{itemize}
        \item Sorpresa: mantiene la atenci\'on
        \item recordemos el gr\'afico \ldots
        \only<2> {
            \item Se lo esperaban?
            \item Hubiera tenido el mismo efecto?
        }
    \end{itemize}
\column{.3\textwidth}
    \begin{center}
        \only<1> {
            \pgfdeclareimage[height=3.5cm]{whoa}{whoa}
            \pgfuseimage{whoa}
        } \only<2> {
            \pgfdeclareimage[height=4cm]{atencion_por_t_ej2}{atencion_por_t_ej}
            \pgfuseimage{atencion_por_t_ej2}
        }
    \end{center}
\end{columns}
\end{frame}

% A todos nos gustan las historias. Son una excelente forma de distraernos
% un rato. Si la presentacion es aburrida, por ejemplo, la audiencia probable-
% mente invente su propia historia, la cual, muy posiblemente, incluya asesinar
% violentamente a quien habla. Para maximizar nuestras posibilidades de supervivencia
% en vez de dejar que la audiencia invente su propia historia es mejor proveerles una.
% Asi sea un ejemplo, como el de Alice y Bob usado para explicar como funcionan
% las claves publicas y privadas, un ejemplo tangencialmente relacionado o un caso
% practico, las historias sirven para darle un hilo narrativo a nuestra presentaci\'on.
% Por nuestra naturaleza humana, no importa que tan tonta sea la historia, vamos a estar
% inclinados a escuchar hasta el final, solo para enterarnos como termina (a menos que
% sea un capitulo de star wars, en ese caso no aplica).
% Si bien no siempre es posible aplicar este tip funciona muy bien de la siguiente forma:
% se presenta un problema, real o no y se desarrolla, junto con la explicaci\'on de
% un tema, la solucion al problema.
\begin{frame}{Usar historias}
\begin{itemize}
    \item Estudiar criptograf\'ia es divertido\ldots gracias a Alice y Bob!
    \item Nos permite darle un hilo narrativo a la charla
    \item Siempre genera intriga
\end{itemize}

\begin{columns}[onlytextwidth]
\column{.5\textwidth}
    \begin{center}
        \pgfdeclareimage[height=3.5cm]{alice}{alice}
        \pgfuseimage{alice}
    \end{center}
\column{.5\textwidth}
    \begin{center}
        \pgfdeclareimage[height=3.5cm]{bob}{bob}
        \pgfuseimage{bob}
    \end{center}
\end{columns}
\end{frame}


% Alguna vez escucharon "yo no lo vote?". Algunos dicen que es por verguenza,
% otros dicen que es porque tenemos cierta tendencia a relacionarnos con
% gente que se nos parece. Esto es por algo muy simple: por lo general nos gusta
% escuchar lo que ya sabemos! (Como dice la vieja frase: dios los cria, ellos se juntan)
% Cuando escuchamos algo que ya sabemos estamos validando nuestro conocimiento
% Una tecnica para dar presentaciones es partir de una base conocida por todos
% y desarrollar partes mas complejas del tema a partir de esta base
\begin{frame}{Es ovbio!}
\textit{ En general, nos gusta escuchar cosas que ya sabemos\ldots } \
\textit{ Por qu\'e?}
\only<2> {
    \begin{itemize}
        \item No es por tontos
        \item Ni es por vagos
        \item Nos gusta tener raz\'on
        \item y validar nuestros conocimientos
    \end{itemize}
}

\only<1> {
    \begin{center}
        \pgfdeclareimage[height=3.5cm]{ovbious}{ovbious}
        \pgfuseimage{ovbious}
    \end{center}
} \only<2> {
    \begin{center}
        \pgfdeclareimage[height=3.5cm]{dumb}{dumb}
        \pgfuseimage{dumb}
    \end{center}
}
\end{frame}


% Dar vueltas sobre lo mismo tarde o temprano cansa. Para evitar esto lo mejor
% es tener un objetivo claro. Tener en claro de que se va a tratar la charla
% nos permite darnos cuenta si estamos cubriendo mucho (y deberiamos hacer dos
% presentaciones en vez de una, una especie de god object pero de charlas) o 
% si tenemos una charla que no tiene fuerza, le falta sopa. Ademas tener
% un objetivo claro permite que la audiencia sepa que va a escuchar, no llevarse
% sorpresas innecesarias. A nadie le gustaria, por ejemplo, que yo anuncie un
% evento sobre gatos, y nos vayamos todos a un gaterio... bueh, mal ejemplo ese,
% pero seguro captan la idea.
% Como transmitimos el objetivo de la charla? Facil, por medio de un resumen. Una
% linea y una imagen, nada mas. Si eso no alcanza para captar la atencion de alguien
% probablemente haya que repensar de que se va a tratar el evento!
\begin{frame}{Zippear la charla}
\textit{ Si hay poca atenci\'on no hay tiempo que perder! }
\begin{columns}[onlytextwidth]
\column{.7\textwidth}
    \begin{itemize}
        \item Hay un objetivo claro?
        \item Es resumible en una frase?
        \item Tiene una imagen caracter\'istica?
    \end{itemize}
\column{.3\textwidth}
    \begin{center}
        \pgfdeclareimage[height=3.5cm]{compress}{compress}
        \pgfuseimage{compress}
    \end{center}
\end{columns}
else \ldots \alert{ goto } 0
\end{frame}

% En la misma linea de la sorpresa y contrapuesto a la idea de repetir cosas ya
% sabidas, los cliffhangers, ademas de ser una forma barata de enganchar a senoras 
% cincuentonas con la novela de las 3, son una ayuda adicional para dar una base
% solida a una charla. La idea de un cliffhanger es dejar al que nos escucha 
% pensando (pude no aplicar para todas las audiencias).
\begin{frame}{Cliffhangers}
\end{frame}

% Cuando el tiempo es poco es importante organizar el contenido segun la curva de
% atencion. Si, como ya estamos terminando eso implica que este es el tip mas
% aburrido de todos los que tenia en la lista de tips.
\begin{frame}{Organizar el contenido}
\end{frame}

\begin{frame}{??}
\end{frame}

\end{document}
