\documentclass{beamer}
% \documentclass[draft]{beamer}

% Vista para impresión
% \usepackage{pgfpages}
% \pgfpagesuselayout{8 on 1}[a4paper,border shrink=5mm]

\mode<presentation>
{
  \usetheme{Frankfurt}
  \useinnertheme{rounded}
  \setbeamerfont{title}{shape=\itshape,family=\rmfamily}
  \setbeamercovered{transparent}
}

\usepackage[english]{babel}
\usepackage[latin1]{inputenc}

\usepackage{times}
\usepackage[T1]{fontenc}

\institute[BoC Tech Talks]{BoC Tech Talks}

\date[] {\today}

\title[SQL from scratch]{SQL from scratch}
\subtitle{}
\author[]{Nicol\'as Brailovsky}

% This is only inserted into the PDF information catalog. Can be left out. 
\subject{}


\pgfdeclareimage[height=5cm]{squirrel_overdose}{squirrel_overdose}
\pgfdeclareimage[height=1cm]{squirrel_hat}{squirrel_hat}
\pgfdeclareimage[height=4cm]{squirrel_hat_big}{squirrel_hat_big}
\logo{\pgfuseimage{squirrel_hat}}

% Delete this, if you do not want the table of contents to pop up at
% the beginning of each subsection:
\AtBeginSubsection[]
{
  \begin{frame}<beamer>{Outline}
    \tableofcontents[currentsection,currentsubsection]
  \end{frame}
}
\AtBeginSection[]
{
  \begin{frame}<beamer>{Outline}
    \tableofcontents[currentsection,currentsubsection]
  \end{frame}
}

\begin{document}

\begin{frame}[plain]
	\begin{center} \pgfuseimage{squirrel_overdose} \end{center}
	\rule{0em}{5pt}

	\setbeamercolor{title}{fg=white,bg=teal}
	\begin{beamercolorbox}[sep=.1cm,shadow=true,rounded=true,center]{title}
		\inserttitle
	\end{beamercolorbox}

	\begin{columns}[onlytextwidth]
	\column{.5\textwidth}
		\begin{center}
			\pgfdeclareimage[height=1cm]{boc}{boc}
			\pgfuseimage{boc}
		\end{center}
	\column{.5\textwidth}
		\begin{center} {\color{orange} \insertinstitute }  \end{center}
	\end{columns}

	\begin{center}
		{\color{orange} 
			\insertauthor \\
			\rule{0em}{8pt}
			\insertdate
		}
	\end{center}
\end{frame}

\begin{frame}
  \tableofcontents
\textit{Oh, yes. Little Bobby tables we call him.} \bigskip
\begin{center}\textit{ -- XKCD} \end{center}
\end{frame}

\section{Introducci\'on}

% SQL tiene sus raices en Prolog, lenguaje declarativo (no procedural).
% Desarrollado en la decada del 70 y popularizado por Relational Software (Oracle!) sobre la base del trabajo de Codd y Companía: la normalización.
% Con el tiempo un lenguaje de progamación fue creciendo un tanto organicamente y se transformo en lo que hoy en dia es TransactSQL o similares.
\begin{frame}{Historia}
	Structured Query Language:
	\begin{itemize}
		\item Es un hijo de\ldots Prolog (Prolog -> Datalog -> SQL)
		\item Es declarativo
		\item Despu\'es se agreg\'o un hack procedural. Se llama PLSQL.
	\end{itemize}
\end{frame}

\begin{frame}{Motores}
SQL es una opci\'on entre muchas \bigskip
\begin{tabular}{ | l | c | r | } \hline
\alert{Big hashmap} & \alert{Object DB} & \alert{RDBMS} \\ \hline
Berkeley DB (Ahora Oracle)&Gemstone&MySQL (Ahora Oracle)\\ \hline
Mongo DB (En serio)&Db4O&PostgreSQL\\ \hline
\end{tabular}
\begin{center}
\pgfdeclareimage[height=3cm]{alvin_y_las_ardillas}{alvin_y_las_ardillas}
\pgfuseimage{alvin_y_las_ardillas}
\end{center}
\end{frame}


\section{SQL en serio}


%  DML es lo que vamos a usar el 99% del tiempo. Son los inserts, 
% deletes y updates. El DDL define la estructura de una tabla, 
% los índices, las PKs, FKs, etc. Vamos a hablar mas que nada del 
% DML porque es una charla orientada al desarrollo (y porque de DDL 
% no se nada, pero no se lo digan a nadie).
\begin{frame}{DML vs DDL}
Hay dos tipos de sentencias SQL:
\begin{center}
\begin{tabular}{ | c | c | } \hline
\alert{DML} & \alert{DDL} \\ \hline
\alert{Data Manipulation Language} & \alert{Data Definition Language} \\ \hline
Consultas a los datos&Metadata, definici\'on de datos \\ \hline
Bastante est\'andar&Rez\'a por tener un buen manual \\ \hline
No debugeable&No debugeable \\ \hline
\end{tabular}
\end{center}
\only<2>{ Vamos a usar DMLs el 99\% del tiempo }
\end{frame}


\begin{frame}{DMLs}
	Hay cuatro tipos de DMLs:
	\begin{columns}[onlytextwidth]
	\column{.5\textwidth}
		\begin{itemize}
			\item DELETE
			\item INSERT
			\item UPDATE
			\item SELECT
		\end{itemize}
	\column{.5\textwidth}
		\pgfdeclareimage[height=4cm]{squirrel-small}{squirrel-small}
		\pgfuseimage{squirrel-small}
	\end{columns}
\end{frame}


\begin{frame}{Deletes}
Es el m\'as simple:
	\begin{columns}[onlytextwidth]
	\column{.8\textwidth}
	\begin{itemize}
		\item DELETE FROM tabla WHERE condicion
	\end{itemize}
	\column{.2\textwidth}
		\pgfdeclareimage[height=4cm]{iceage}{iceage}
		\pgfuseimage{iceage}
	\end{columns}
\alert{Tip: primero el where, despu\'es el delete.} \only<2>{\alert{Si, por experiencia.} }
\end{frame}


\begin{frame}[t]{Where}
Los where tambi\'en son simples
\begin{itemize}
	\item WHERE (condiciones)
	\item Las condiciones son booleans
	\item WHERE x=1 AND (y=2 OR z=3)
\end{itemize}
\textit{Tip:} El elemento neutro ahorra c\'odigo.\only<2>{ Un ejemplo? }
\only<3,4>{
		\begin{itemize}
			\item WHERE false OR (x=4 AND z=2) OR (x=42 AND z=42)
			\item WHERE user\_id IN (NULL, 1, 2, 3);
		\end{itemize}
	}
\only<4>{
		\begin{center}
		\pgfdeclareimage[height=3cm]{taliban}{taliban}
		\pgfuseimage{taliban}
		\end{center}
	}
\end{frame}


\begin{frame}{Updates}
Es el segundo m\'as simple:
	\begin{columns}[onlytextwidth]
	\column{.8\textwidth}
	\begin{itemize}
		\item UPDATE tabla SET key=val WHERE condicion
	\end{itemize}
	\column{.2\textwidth}
		\pgfdeclareimage[height=4cm]{iceage}{iceage}
		\pgfuseimage{iceage}
	\end{columns}
\alert{Tambi\'en sirve: primero el where, despu\'es el delete.}
\end{frame}


\begin{frame}{Insert}
Es el tercero m\'as simple (?). Ahora hay algunas variaciones:
\begin{itemize}
	\item INSERT INTO tabla (col1, col2) VALUES (val1, val2); \\
		\only<2,3,4>{
			\begin{itemize}
				\item No importa el orden
				\item Puede estar incompleto
			\end{itemize}
		}
	\item INSERT INTO tabla VALUES (val1, val2); \\
		\only<3,4>{
			\begin{itemize}
				\item Importa el orden
				\item No puede estar incompleto
			\end{itemize}
		}
	\item INSERT INTO tabla SELECT (WTF!?) \\
		\only<4>{
			\begin{itemize}
				\item Ehh... lo vemos despu\'es
			\end{itemize}
		}
\end{itemize}
\only<1>{\textit{Esta vez no hay tip loco}}
\end{frame}

\begin{frame}{Insert II}
Otra variaci\'on m\'as:
\begin{semiverbatim}
	INSERT INTO tabla (col1, col2) VALUES
							(val1, val2), 
							(bal1, bal2), 
							(wal1, wal2);
\end{semiverbatim}
\only<1>{\textit{Disminuye la carga en la DB!}}
\end{frame}


\begin{frame}{Select}
\begin{center}Es un bardo. Posta. Sirve para traer datos. Si, en serio. \end{center}
\only<1>{En su forma APB: }
\only<2>{O en su forma Oracle APB: }
\only<3>{Usando una funci\'on}
\begin{itemize}
	\item SELECT 42
	\only<2,3>{\item SELECT 42 FROM DUAL (Gracias Oracle!)}
	\only<3>{\item SELECT NOW()}
\end{itemize}
\begin{center} \pgfuseimage{squirrel_overdose} \end{center}
\end{frame}


\begin{frame}{Select de verdad}
\begin{semiverbatim}SELECT * FROM users\end{semiverbatim}
\only<2>{
	Si tenemos estos datos...
	\begin{semiverbatim}INSERT INTO USERS (name, age) VALUES\end{semiverbatim}
	\begin{semiverbatim}('Bart',10), ('Lisa',8),\end{semiverbatim}
	\begin{semiverbatim}('Maggie',2), ('Carl', 40);\end{semiverbatim}
	\begin{semiverbatim}INSERT INTO RELATIONSHIPS (user1\_id, user2\_id, related) \end{semiverbatim}
	\begin{semiverbatim}VALUES (1, 2, 'Hermanos'), (1,3, 'Hermanos'), \end{semiverbatim}
	\begin{semiverbatim}(2,1, 'Hermanos'), (2,3, 'Hermanos'),\end{semiverbatim}
	\begin{semiverbatim}(3,1, 'Hermanos'),(3,2, 'Hermanos');\end{semiverbatim}
}
\end{frame}

\begin{frame}{Select de verdad II}
\begin{columns}[onlytextwidth]
\column{.5\textwidth}
	\begin{semiverbatim}SELECT * \end{semiverbatim}
	\begin{semiverbatim}FROM users\end{semiverbatim}
	\begin{semiverbatim}WHERE age < 10\end{semiverbatim}
\column{.5\textwidth}
	\begin{center}
	\pgfdeclareimage[height=2cm]{q1}{q1}
	\pgfuseimage{q1}
	\end{center}
\end{columns}
\end{frame}


\section{Ahora s\'i la complicamos}



\begin{frame}[t]{Select de verdad III}
\textit{C\'omo encontramos todos los usuarios y sus hermanos?}
\only<2>{
	\begin{semiverbatim}SELECT * \end{semiverbatim}
	\begin{semiverbatim}FROM users \alert{Usr1}\end{semiverbatim}
	\begin{semiverbatim}INNER JOIN \alert{relationships Rel}\end{semiverbatim}
	\begin{semiverbatim} \alert{ ON Rel.user1\_id = Usr1.id}\end{semiverbatim}
	\begin{semiverbatim} \alert{WHERE related = 'Hermanos'}\end{semiverbatim}
}\only<3>{
	\begin{center}
	\pgfdeclareimage[height=4cm]{q2}{q2}
	\pgfuseimage{q2}
	\end{center}
}
\only<3>{\textit{C\'omo nos traemos el nombre del otro hermano?}}
\end{frame}


\begin{frame}{Select de verdad IV}
\begin{itemize}
	\item \textit{C\'omo encontramos todos los usuarios y sus hermanos?}
	\item \textit{C\'omo nos traemos el nombre del otro hermano?}
\end{itemize}
\only<2>{
	\begin{semiverbatim}SELECT Usr1.name, Usr2.name \end{semiverbatim}
	\begin{semiverbatim}FROM users Usr1\end{semiverbatim}
	\begin{semiverbatim}INNER JOIN relationships Rel\end{semiverbatim}
	\begin{semiverbatim}ON Rel.user1\_id = Usr1.id\end{semiverbatim}
	\begin{semiverbatim}\alert{INNER JOIN users Usr2}\end{semiverbatim}
	\begin{semiverbatim} \alert{ ON Rel.user2\_id = Usr2.id}\end{semiverbatim}
	\begin{semiverbatim}WHERE related = 'Hermanos'\end{semiverbatim}
}\only<3>{
	\begin{center}
	\pgfdeclareimage[height=4cm]{q3}{q3}
	\pgfuseimage{q3}
	\end{center}
}
\end{frame}


\begin{frame}{Select de verdad IV'}
	\begin{center}
		\textquestiondown Qu\'e pasa si yo quer\'ia TODOS los usuarios? \\
		\textquestiondown A d\'onde fue Carl? \\
		\pgfdeclareimage[height=5cm]{fail}{fail}
		\pgfuseimage{fail}
	\end{center}
\end{frame}


\begin{frame}{Select de verdad IV}
\only<1>{
	\begin{semiverbatim}SELECT Usr1.name, Usr2.name \end{semiverbatim}
	\begin{semiverbatim}FROM users Usr1\end{semiverbatim}
	\begin{semiverbatim}\alert{LEFT} JOIN relationships Rel\end{semiverbatim}
	\begin{semiverbatim}ON Rel.user1\_id = Usr1.id\end{semiverbatim}
	\begin{semiverbatim}\alert{LEFT} JOIN users Usr2 ON Rel.user2\_id = Usr2.id\end{semiverbatim}
}\only<2>{
	\begin{center}
	\pgfdeclareimage[height=4cm]{q4}{q4}
	\pgfuseimage{q4}
	\end{center}
}
\end{frame}

\begin{frame}{Select de verdad V}
	\begin{center}
		\textquestiondown Y si quiero traer s\'olo cu\'antos hermanos tienen?
		\pgfdeclareimage[height=5cm]{carl}{carl}
		\pgfuseimage{carl}
	\end{center}
\end{frame}

\begin{frame}{Select de verdad V}
\only<1>{
	\begin{semiverbatim}SELECT Usr1.name, \alert{count(1) AS Hermanos} \end{semiverbatim}
	\begin{semiverbatim}FROM users Usr1\end{semiverbatim}
	\begin{semiverbatim}LEFT JOIN relationships Rel\end{semiverbatim}
	\begin{semiverbatim}ON Rel.user1\_id = Usr1.id\end{semiverbatim}
	\begin{semiverbatim}LEFT JOIN users Usr2 ON Rel.user2\_id = Usr2.id\end{semiverbatim}
	\begin{semiverbatim}\alert{GROUP BY Usr1.id}\end{semiverbatim}
}\only<2>{
	\begin{center}
	\pgfdeclareimage[height=4cm]{q5}{q5}
	\pgfuseimage{q5}
	\end{center}
}
\textit{Cuidado, COUNT+LEFT JOIN = cosas raras}
\end{frame}


\begin{frame}{Select de verdad VI}
	\begin{center}
		El GROUP tiene su WHERE, pero se llama HAVING
		\pgfdeclareimage[height=5cm]{squirrel_overdose}{squirrel_overdose}
		\pgfuseimage{squirrel_overdose}
	\end{center}
\end{frame}

\begin{frame}{Select de verdad VI}
\begin{semiverbatim}SELECT Usr1.name, COUNT(1) AS Hermanos \end{semiverbatim}
\begin{semiverbatim}FROM users Usr1\end{semiverbatim}
\begin{semiverbatim}LEFT JOIN relationships Rel\end{semiverbatim}
\begin{semiverbatim}ON Rel.user1\_id = Usr1.id\end{semiverbatim}
\begin{semiverbatim}LEFT JOIN users Usr2 ON Rel.user2\_id = Usr2.id\end{semiverbatim}
\begin{semiverbatim}GROUP BY Usr1.id\end{semiverbatim}
\begin{semiverbatim}\alert{HAVING COUNT(1) > 1}\end{semiverbatim}
\end{frame}

\begin{frame}{Select de verdad VII, s\'olo para complicarla}
\begin{semiverbatim}SELECT Usr1.name, COUNT(1) AS Hermanos \end{semiverbatim}
\begin{semiverbatim}FROM users Usr1\end{semiverbatim}
\begin{semiverbatim}LEFT JOIN relationships Rel\end{semiverbatim}
\begin{semiverbatim}ON Rel.user1\_id = Usr1.id\end{semiverbatim}
\begin{semiverbatim}LEFT JOIN users Usr2 ON Rel.user2\_id = Usr2.id\end{semiverbatim}
\begin{semiverbatim}GROUP BY Usr1.id\end{semiverbatim}
\begin{semiverbatim}HAVING COUNT(1) > 1\end{semiverbatim}
\begin{semiverbatim}\alert{ORDER BY Usr1.name LIMIT 3}\end{semiverbatim}
\end{frame}

\section{Misc}
\begin{frame}{COALESCE, NVL, IFNULL}
\begin{itemize}
	\item Son lo mismo pero cambia seg\'un el motor
	\item Es un if trucho
	\item IFNULL(A, B) <=> A!=NULL? A : B
\end{itemize}
\end{frame}

\begin{frame}{Unions}
\end{frame}

\begin{frame}{Joins raros}
\end{frame}

\begin{frame}{Subqueries}
\end{frame}

\end{document}
