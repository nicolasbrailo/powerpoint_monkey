\section{Mocking}

\begin{frame}[t]{Mocking: seguimos el ejemplo}
Sigamos el ejemplo de DI. \textquestiondown Podemos mejoralo mas?
\begin{itemize}
	\item \textquestiondown Qu\'e estamos testeando?
	\item \textquestiondown Queremos testear el objeto "Cuenta"?
	\item \textquestiondown Y si Cuenta a\'un no est\'a implementado?
	\item \textquestiondown O tiene muchos muchos bugs?
	\item \textquestiondown O es muy lento? (P. ej: accede a la BD)
\end{itemize}
\bigskip
\begin{beamercolorbox}[sep=.1cm,shadow=true,rounded=true,center]{title}
	Mocking es la soluci\'on!
\end{beamercolorbox}
\end{frame}

\begin{frame}{Mocking}
\textquestiondown Pero qu\'e es "mocking"?
\bigskip
\begin{columns}[onlytextwidth]
\column{.7\textwidth}
	\begin{itemize}
		\item t\'ecnica para testear
		\item un mock es un objeto "dummy"
		\item son "fakes" de partes del sistema 
		\item definen su comportamiento "on the fly"
		\item usan la interfaz de un objeto real
	\end{itemize}
\column{.3\textwidth}
	\pgfdeclareimage[height=3cm]{img/misc/mockingbird}{img/misc/mockingbird}
	\pgfuseimage{img/misc/mockingbird}
\end{columns}
\end{frame}

%%%%%%%%%%%%%%%%%%%%%%%%%%%%%%%%%%%%%%%%%%%%%%%%%%
\subsection{Herencia, polimorfismo y Mocking}

\begin{frame}[t]{C\'omo funciona}
\begin{itemize}
	\item Se basa en el polimorfismo
	\item El Objeto testeable usa un objeto "mockeable"
	\item El mock hereda de una interfaz com\'un al objeto real
	\item Al objeto testeable no le "importa" cual de los dos usa
\end{itemize}
\bigskip
\begin{center}
	\pgfdeclareimage[height=3cm]{ejemplo1/mocking}{ejemplo1/mocking}
	\pgfuseimage{ejemplo1/mocking}
\end{center}
\end{frame}

\begin{frame}[t]{C\'omo funciona}
\begin{itemize}
	\item El objeto testeado no sabe si es una implementaci\'on real
	\item Tampoco importa, el mock usa la interfaz conocida
	\item Es un concepto similar al de usar varias implementaciones
\end{itemize}
\bigskip
\begin{center}
	\pgfdeclareimage[height=3.5cm]{ejemplo1/mocking_example}{ejemplo1/mocking_example}
	\pgfuseimage{ejemplo1/mocking_example}
\end{center}
\end{frame}

%%%%%%%%%%%%%%%%%%%%%%%%%%%%%%%%%%%%%%%%%%%%%%%%%%
\subsection{DI y Mocking}
\begin{frame}[t]{Por qu\'e DI + Mocking}
\bigskip

\begin{columns}[onlytextwidth]
\column{.7\textwidth}
	\begin{itemize}
		\item DI obliga a separar componentes
		\item Para testear hace falta aislar m\'odulos
		\item Aislar m\'odulo $\implies$ sacar dependencias
		\item Mocking permite simular otros m\'odulos

		\item Ergo \ldots
	\end{itemize}
\column{.3\textwidth}
	\begin{flushright}
		\pgfdeclareimage[height=2cm]{img/misc/tester}{img/misc/tester}
		\pgfuseimage{img/misc/tester}
	\end{flushright}
\end{columns}
\bigskip
\bigskip
\begin{beamercolorbox}[sep=.1cm,shadow=true,rounded=true,center]{title}
	Mocking + DI r0ck\$!!!!!11uno1
\end{beamercolorbox}
\end{frame}

%%%%%%%%%%%%%%%%%%%%%%%%%%%%%%%%%%%%%%%%%%%%%%%%%%
\subsection{Ejemplos}

\begin{frame}{Mocking: seguimos el ejemplo II}
	Veamos como queda el ejemplo
	\section{Mocking}

\begin{frame}[t]{Mocking: seguimos el ejemplo}
Sigamos el ejemplo de DI. \textquestiondown Podemos mejoralo mas?
\begin{itemize}
	\item \textquestiondown Qu\'e estamos testeando?
	\item \textquestiondown Queremos testear el objeto "Cuenta"?
	\item \textquestiondown Y si Cuenta a\'un no est\'a implementado?
	\item \textquestiondown O tiene muchos muchos bugs?
	\item \textquestiondown O es muy lento? (P. ej: accede a la BD)
\end{itemize}
\bigskip
\begin{beamercolorbox}[sep=.1cm,shadow=true,rounded=true,center]{title}
	Mocking es la soluci\'on!
\end{beamercolorbox}
\end{frame}

\begin{frame}{Mocking}
\textquestiondown Pero qu\'e es "mocking"?
\bigskip
\begin{columns}[onlytextwidth]
\column{.7\textwidth}
	\begin{itemize}
		\item t\'ecnica para testear
		\item un mock es un objeto "dummy"
		\item son "fakes" de partes del sistema 
		\item definen su comportamiento "on the fly"
		\item usan la interfaz de un objeto real
	\end{itemize}
\column{.3\textwidth}
	\pgfdeclareimage[height=3cm]{img/misc/mockingbird}{img/misc/mockingbird}
	\pgfuseimage{img/misc/mockingbird}
\end{columns}
\end{frame}

%%%%%%%%%%%%%%%%%%%%%%%%%%%%%%%%%%%%%%%%%%%%%%%%%%
\subsection{Herencia, polimorfismo y Mocking}

\begin{frame}[t]{C\'omo funciona}
\begin{itemize}
	\item Se basa en el polimorfismo
	\item El Objeto testeable usa un objeto "mockeable"
	\item El mock hereda de una interfaz com\'un al objeto real
	\item Al objeto testeable no le "importa" cual de los dos usa
\end{itemize}
\bigskip
\begin{center}
	\pgfdeclareimage[height=3cm]{ejemplo1/mocking}{ejemplo1/mocking}
	\pgfuseimage{ejemplo1/mocking}
\end{center}
\end{frame}

\begin{frame}[t]{C\'omo funciona}
\begin{itemize}
	\item El objeto testeado no sabe si es una implementaci\'on real
	\item Tampoco importa, el mock usa la interfaz conocida
	\item Es un concepto similar al de usar varias implementaciones
\end{itemize}
\bigskip
\begin{center}
	\pgfdeclareimage[height=3.5cm]{ejemplo1/mocking_example}{ejemplo1/mocking_example}
	\pgfuseimage{ejemplo1/mocking_example}
\end{center}
\end{frame}

%%%%%%%%%%%%%%%%%%%%%%%%%%%%%%%%%%%%%%%%%%%%%%%%%%
\subsection{DI y Mocking}
\begin{frame}[t]{Por qu\'e DI + Mocking}
\bigskip

\begin{columns}[onlytextwidth]
\column{.7\textwidth}
	\begin{itemize}
		\item DI obliga a separar componentes
		\item Para testear hace falta aislar m\'odulos
		\item Aislar m\'odulo $\implies$ sacar dependencias
		\item Mocking permite simular otros m\'odulos

		\item Ergo \ldots
	\end{itemize}
\column{.3\textwidth}
	\begin{flushright}
		\pgfdeclareimage[height=2cm]{img/misc/tester}{img/misc/tester}
		\pgfuseimage{img/misc/tester}
	\end{flushright}
\end{columns}
\bigskip
\bigskip
\begin{beamercolorbox}[sep=.1cm,shadow=true,rounded=true,center]{title}
	Mocking + DI r0ck\$!!!!!11uno1
\end{beamercolorbox}
\end{frame}

%%%%%%%%%%%%%%%%%%%%%%%%%%%%%%%%%%%%%%%%%%%%%%%%%%
\subsection{Ejemplos}

\begin{frame}{Mocking: seguimos el ejemplo II}
	Veamos como queda el ejemplo
	\section{Mocking}

\begin{frame}[t]{Mocking: seguimos el ejemplo}
Sigamos el ejemplo de DI. \textquestiondown Podemos mejoralo mas?
\begin{itemize}
	\item \textquestiondown Qu\'e estamos testeando?
	\item \textquestiondown Queremos testear el objeto "Cuenta"?
	\item \textquestiondown Y si Cuenta a\'un no est\'a implementado?
	\item \textquestiondown O tiene muchos muchos bugs?
	\item \textquestiondown O es muy lento? (P. ej: accede a la BD)
\end{itemize}
\bigskip
\begin{beamercolorbox}[sep=.1cm,shadow=true,rounded=true,center]{title}
	Mocking es la soluci\'on!
\end{beamercolorbox}
\end{frame}

\begin{frame}{Mocking}
\textquestiondown Pero qu\'e es "mocking"?
\bigskip
\begin{columns}[onlytextwidth]
\column{.7\textwidth}
	\begin{itemize}
		\item t\'ecnica para testear
		\item un mock es un objeto "dummy"
		\item son "fakes" de partes del sistema 
		\item definen su comportamiento "on the fly"
		\item usan la interfaz de un objeto real
	\end{itemize}
\column{.3\textwidth}
	\pgfdeclareimage[height=3cm]{img/misc/mockingbird}{img/misc/mockingbird}
	\pgfuseimage{img/misc/mockingbird}
\end{columns}
\end{frame}

%%%%%%%%%%%%%%%%%%%%%%%%%%%%%%%%%%%%%%%%%%%%%%%%%%
\subsection{Herencia, polimorfismo y Mocking}

\begin{frame}[t]{C\'omo funciona}
\begin{itemize}
	\item Se basa en el polimorfismo
	\item El Objeto testeable usa un objeto "mockeable"
	\item El mock hereda de una interfaz com\'un al objeto real
	\item Al objeto testeable no le "importa" cual de los dos usa
\end{itemize}
\bigskip
\begin{center}
	\pgfdeclareimage[height=3cm]{ejemplo1/mocking}{ejemplo1/mocking}
	\pgfuseimage{ejemplo1/mocking}
\end{center}
\end{frame}

\begin{frame}[t]{C\'omo funciona}
\begin{itemize}
	\item El objeto testeado no sabe si es una implementaci\'on real
	\item Tampoco importa, el mock usa la interfaz conocida
	\item Es un concepto similar al de usar varias implementaciones
\end{itemize}
\bigskip
\begin{center}
	\pgfdeclareimage[height=3.5cm]{ejemplo1/mocking_example}{ejemplo1/mocking_example}
	\pgfuseimage{ejemplo1/mocking_example}
\end{center}
\end{frame}

%%%%%%%%%%%%%%%%%%%%%%%%%%%%%%%%%%%%%%%%%%%%%%%%%%
\subsection{DI y Mocking}
\begin{frame}[t]{Por qu\'e DI + Mocking}
\bigskip

\begin{columns}[onlytextwidth]
\column{.7\textwidth}
	\begin{itemize}
		\item DI obliga a separar componentes
		\item Para testear hace falta aislar m\'odulos
		\item Aislar m\'odulo $\implies$ sacar dependencias
		\item Mocking permite simular otros m\'odulos

		\item Ergo \ldots
	\end{itemize}
\column{.3\textwidth}
	\begin{flushright}
		\pgfdeclareimage[height=2cm]{img/misc/tester}{img/misc/tester}
		\pgfuseimage{img/misc/tester}
	\end{flushright}
\end{columns}
\bigskip
\bigskip
\begin{beamercolorbox}[sep=.1cm,shadow=true,rounded=true,center]{title}
	Mocking + DI r0ck\$!!!!!11uno1
\end{beamercolorbox}
\end{frame}

%%%%%%%%%%%%%%%%%%%%%%%%%%%%%%%%%%%%%%%%%%%%%%%%%%
\subsection{Ejemplos}

\begin{frame}{Mocking: seguimos el ejemplo II}
	Veamos como queda el ejemplo
	\section{Mocking}

\begin{frame}[t]{Mocking: seguimos el ejemplo}
Sigamos el ejemplo de DI. \textquestiondown Podemos mejoralo mas?
\begin{itemize}
	\item \textquestiondown Qu\'e estamos testeando?
	\item \textquestiondown Queremos testear el objeto "Cuenta"?
	\item \textquestiondown Y si Cuenta a\'un no est\'a implementado?
	\item \textquestiondown O tiene muchos muchos bugs?
	\item \textquestiondown O es muy lento? (P. ej: accede a la BD)
\end{itemize}
\bigskip
\begin{beamercolorbox}[sep=.1cm,shadow=true,rounded=true,center]{title}
	Mocking es la soluci\'on!
\end{beamercolorbox}
\end{frame}

\begin{frame}{Mocking}
\textquestiondown Pero qu\'e es "mocking"?
\bigskip
\begin{columns}[onlytextwidth]
\column{.7\textwidth}
	\begin{itemize}
		\item t\'ecnica para testear
		\item un mock es un objeto "dummy"
		\item son "fakes" de partes del sistema 
		\item definen su comportamiento "on the fly"
		\item usan la interfaz de un objeto real
	\end{itemize}
\column{.3\textwidth}
	\pgfdeclareimage[height=3cm]{img/misc/mockingbird}{img/misc/mockingbird}
	\pgfuseimage{img/misc/mockingbird}
\end{columns}
\end{frame}

%%%%%%%%%%%%%%%%%%%%%%%%%%%%%%%%%%%%%%%%%%%%%%%%%%
\subsection{Herencia, polimorfismo y Mocking}

\begin{frame}[t]{C\'omo funciona}
\begin{itemize}
	\item Se basa en el polimorfismo
	\item El Objeto testeable usa un objeto "mockeable"
	\item El mock hereda de una interfaz com\'un al objeto real
	\item Al objeto testeable no le "importa" cual de los dos usa
\end{itemize}
\bigskip
\begin{center}
	\pgfdeclareimage[height=3cm]{ejemplo1/mocking}{ejemplo1/mocking}
	\pgfuseimage{ejemplo1/mocking}
\end{center}
\end{frame}

\begin{frame}[t]{C\'omo funciona}
\begin{itemize}
	\item El objeto testeado no sabe si es una implementaci\'on real
	\item Tampoco importa, el mock usa la interfaz conocida
	\item Es un concepto similar al de usar varias implementaciones
\end{itemize}
\bigskip
\begin{center}
	\pgfdeclareimage[height=3.5cm]{ejemplo1/mocking_example}{ejemplo1/mocking_example}
	\pgfuseimage{ejemplo1/mocking_example}
\end{center}
\end{frame}

%%%%%%%%%%%%%%%%%%%%%%%%%%%%%%%%%%%%%%%%%%%%%%%%%%
\subsection{DI y Mocking}
\begin{frame}[t]{Por qu\'e DI + Mocking}
\bigskip

\begin{columns}[onlytextwidth]
\column{.7\textwidth}
	\begin{itemize}
		\item DI obliga a separar componentes
		\item Para testear hace falta aislar m\'odulos
		\item Aislar m\'odulo $\implies$ sacar dependencias
		\item Mocking permite simular otros m\'odulos

		\item Ergo \ldots
	\end{itemize}
\column{.3\textwidth}
	\begin{flushright}
		\pgfdeclareimage[height=2cm]{img/misc/tester}{img/misc/tester}
		\pgfuseimage{img/misc/tester}
	\end{flushright}
\end{columns}
\bigskip
\bigskip
\begin{beamercolorbox}[sep=.1cm,shadow=true,rounded=true,center]{title}
	Mocking + DI r0ck\$!!!!!11uno1
\end{beamercolorbox}
\end{frame}

%%%%%%%%%%%%%%%%%%%%%%%%%%%%%%%%%%%%%%%%%%%%%%%%%%
\subsection{Ejemplos}

\begin{frame}{Mocking: seguimos el ejemplo II}
	Veamos como queda el ejemplo
	\include{code/mocking}
\end{frame}

\begin{frame}{Mocking: ejemplo explicado}
Algunos detalles del ejemplo:
\include{code/mocking_detail}
\begin{itemize}
	\item El comportamiento se define con el mismo test
	\item Se define un set de "expectations"
	\item Las expectations son tests \\
		\begin{itemize}
			\item cada una debe cumplirse
			\item verifican el comportamiento del objeto
		\end{itemize}
	\item Usando DI podemos inyectar una cuenta falsa
	\item \ldots ser\'ia posible de no usar DI?
\end{itemize}
\end{frame}

%%%%%%%%%%%%%%%%%%%%%%%%%%%%%%%%%%%%%%%%%%%%%%%%%%

\end{frame}

\begin{frame}{Mocking: ejemplo explicado}
Algunos detalles del ejemplo:
\include{code/mocking_detail}
\begin{itemize}
	\item El comportamiento se define con el mismo test
	\item Se define un set de "expectations"
	\item Las expectations son tests \\
		\begin{itemize}
			\item cada una debe cumplirse
			\item verifican el comportamiento del objeto
		\end{itemize}
	\item Usando DI podemos inyectar una cuenta falsa
	\item \ldots ser\'ia posible de no usar DI?
\end{itemize}
\end{frame}

%%%%%%%%%%%%%%%%%%%%%%%%%%%%%%%%%%%%%%%%%%%%%%%%%%

\end{frame}

\begin{frame}{Mocking: ejemplo explicado}
Algunos detalles del ejemplo:
\include{code/mocking_detail}
\begin{itemize}
	\item El comportamiento se define con el mismo test
	\item Se define un set de "expectations"
	\item Las expectations son tests \\
		\begin{itemize}
			\item cada una debe cumplirse
			\item verifican el comportamiento del objeto
		\end{itemize}
	\item Usando DI podemos inyectar una cuenta falsa
	\item \ldots ser\'ia posible de no usar DI?
\end{itemize}
\end{frame}

%%%%%%%%%%%%%%%%%%%%%%%%%%%%%%%%%%%%%%%%%%%%%%%%%%

\end{frame}

\begin{frame}{Mocking: ejemplo explicado}
Algunos detalles del ejemplo:
\include{code/mocking_detail}
\begin{itemize}
	\item El comportamiento se define con el mismo test
	\item Se define un set de "expectations"
	\item Las expectations son tests \\
		\begin{itemize}
			\item cada una debe cumplirse
			\item verifican el comportamiento del objeto
		\end{itemize}
	\item Usando DI podemos inyectar una cuenta falsa
	\item \ldots ser\'ia posible de no usar DI?
\end{itemize}
\end{frame}

%%%%%%%%%%%%%%%%%%%%%%%%%%%%%%%%%%%%%%%%%%%%%%%%%%
