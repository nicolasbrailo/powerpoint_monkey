\section{Google Test}

\begin{frame}[t]{No m\'as asserts}
\begin{center}
	\pgfdeclareimage[height=1cm]{img/logos/google_test}{img/logos/google_test}
	\pgfuseimage{img/logos/google_test}
\end{center}
\textit{ \textquestiondown Por qu\'e un framework de testing? }
\bigskip
\begin{itemize}
	\item Nos olvidamos del scaffolding
	\item Ayuda a crear tests independientes y repetibles
	\item Imprime mas info de debug que un assert
	\item Provee herramientas para distintos tipos de tests
\end{itemize}
\bigskip
RTFM @ \url{http://code.google.com/p/googletest}
\end{frame}

\begin{frame}[t]{GoogleTest: Conceptos B\'asicos}
\begin{flushright}
	RTFM @ \href{http://code.google.com/p/googletest/wiki/GoogleTestAdvancedGuide}{GoogleTestAdvancedGuide}
\end{flushright}
\begin{beamercolorbox}[sep=.1cm,shadow=true,rounded=true,center]{title}
	Estructura 
\end{beamercolorbox}
\begin{itemize}
	\item Cada test por separado, con la macro TEST
	\item El test tiene un nombre y un grupo
	\item Hay "expectations" en distintos puntos
	\item Si requiere estado $\implies$ TEST\_F (fixture)
\end{itemize}
\bigskip
\begin{beamercolorbox}[sep=.1cm,shadow=true,rounded=true,center]{title}
	Expectations / Assertions: Lo mas simple
\end{beamercolorbox}
\begin{itemize}
	\item assert() $\implies$ ASSERT\_* y EXPECT\_*
	\item \{ASSERT, EXPECT\}\_\{TRUE, FALSE, EQ, LT, STREQ\}
	\item EXPECT\_* $\implies$ no fatal
	\item ASSERT\_* $\implies$ fatal (interrumpe el test)
\end{itemize}
\bigskip
\end{frame}


\begin{frame}[shrink=2]{Un ejemplo y su test}
\include{code/protocol1_ut2}
\end{frame}


\begin{frame}[shrink=2]{Un ejemplo y su test}
\include{code/protocol1_ut2_b}
\end{frame}


\begin{frame}{Un ejemplo y su test}
\textit{Est\'a un poco mejor, pero todav\'ia\ldots}
\begin{itemize}
	\item Hay mucho scaffolding
	\item Es fr\'agil
	\item Expectations en todos lados $\implies$ test poco claro
\end{itemize}
\bigskip
\textit{\textquestiondown Qu\'e ganamos?}
\begin{itemize}
	\item Menos c\'odigo (Linkear con -lgtest\_main)
	\item Mejores mensajes de error
\end{itemize}
\bigskip
\alert{Veamos como se ve en funcionamiento\ldots}
\end{frame}


\begin{frame}[fragile]{Test run del test: Todo OK}
\begin{semiverbatim}
Running main() from gtest_main.cc
{\color[rgb]{0,1,0}[==========]} Running 1 test from 1 test case.
{\color[rgb]{0,1,0}[---------------]} Global test environment set-up.
{\color[rgb]{0,1,0}[---------------]} 1 test from Handshake
{\color[rgb]{0,1,0}[ RUN      ]} Handshake.TxOK
{\color[rgb]{0,1,0}[       OK ]} Handshake.TxOK
{\color[rgb]{0,1,0}[---------------]} Global test environment tear-down
{\color[rgb]{0,1,0}[==========]} 1 test from 1 test case ran.
{\color[rgb]{0,1,0}[  PASSED  ]} 1 test.
\end{semiverbatim}
\end{frame}

\begin{frame}[shrink=2,fragile]{Test run del test: Una falla}
\begin{semiverbatim}
Running main() from gtest_main.cc
{\color[rgb]{0,1,0}[==========]} Running 1 test from 1 test case.
{\color[rgb]{0,1,0}[---------------]} Global test environment set-up.
{\color[rgb]{0,1,0}[---------------]} 1 test from Handshake
{\color[rgb]{0,1,0}[ RUN      ]} Handshake.TxOK
test1.cpp:4: Failure
Value of: "Adios mundo cruel"
Expected: (char*)msg
Which is: "Hola mundo"
test1.cpp:20: Failure
Value of: errs
  Actual: 1
Expected: 0
{\color[rgb]{1,0,0}[  FAILED  ]} Handshake.TxOK
{\color[rgb]{0,1,0}[---------------]} Global test environment tear-down
{\color[rgb]{0,1,0}[==========]} 1 test from 1 test case ran.
{\color[rgb]{0,1,0}[  PASSED  ]} 0 tests.
{\color[rgb]{1,0,0}[  FAILED  ]} 1 test, listed below:
{\color[rgb]{1,0,0}[  FAILED  ]} Handshake.TxOK

 1 FAILED TEST
\end{semiverbatim}
\end{frame}


\begin{frame}[t]{Unit Test: The Revenge}
Sabiendo de que se trata gTest\ldots \textit{ Mejoremos un poco el test }
\bigskip
\begin{itemize}
	\item \textquestiondown No molesta el c\'odigo repetido?
	\item \textquestiondown No es feo crear a mano el estado previo al test?
\end{itemize}

\bigskip
\begin{beamercolorbox}[sep=.1cm,shadow=true,rounded=true,center]{title}
	Reduzca el nivel de stress, use Fixtures!
\end{beamercolorbox}

\begin{textblock*}{30mm}(15mm,12mm)
	\pgfdeclareimage[height=3cm]{img/misc/revenge}{img/misc/revenge}
	\pgfuseimage{img/misc/revenge}
\end{textblock*}
\end{frame}


\begin{frame}[shrink=2]{As\'i queda el Fixture}
\include{code/protocol1_ut3}
\end{frame}


\begin{frame}[shrink=2]{As\'i quedan los tests}
\include{code/protocol1_ut3_b}
\end{frame}

\begin{frame}{Google Test: algunos tips}
\begin{itemize}
	\item En EXPECT\_* usar primero el valor y luego la variable \\
				\textit{ Los mensajes de error ser\'an mas claros }
	\item Usar EXPECT\_THROW( \{ \ldots \}, ExcpetionType ) \\
				\textit{ EXPECT\_THROW es muy \'util pero no olvidar las llaves }
	\item Los death tests pueden servir para algo \\
				\textit{ Un death test forkea el proceso y espera que salga el hijo }
	\item No usar especificadores de excepciones \\
				\textit{ Hacen desastres junto con los templates. Son feas anyway.}
	\item No usar como nombre del test el de la clase \\
				\textit{ Pueden ocurrir cosas bizarras y es mas dif\'icil usar GDB }
\end{itemize}
\pgfdeclareimage[height=2cm]{img/misc/tips}{img/misc/tips}
\pgfuseimage{img/misc/tips}
\end{frame}


\begin{frame}[t]{Todav\'ia no terminamos\ldots}
\bigskip
\begin{itemize}
	\item El test sigue siendo fr\'agil
	\item Crear un socket puede ser lento
	\item El c\'odigo es feo y quedan partes sin testear \\
		\begin{itemize}
			\item \textquestiondown C\'omo testeamos que pasa si falla el socket?
			\item \textquestiondown C\'omo testeamos que el comando se env\'ie bien?
		\end{itemize}
\end{itemize}
\bigskip
\begin{beamercolorbox}[sep=.1cm,shadow=true,rounded=true,center]{title}
	Por suerte todav\'ia nos queda ver Google Mock!
\end{beamercolorbox}

\begin{textblock*}{30mm}(90mm,-50mm)
	\pgfdeclareimage[height=2cm]{img/misc/mockingbird}{img/misc/mockingbird}
	\pgfuseimage{img/misc/mockingbird}
\end{textblock*}
\end{frame}

