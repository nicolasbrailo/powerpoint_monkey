\section{Google Mock}


\begin{frame}[t]{Mocking}
\begin{center}
	\pgfdeclareimage[height=1cm]{img/logos/google_mock}{img/logos/google_mock}
	\pgfuseimage{img/logos/google_mock}
\end{center}

\only<1> {
	\textit{ \textquestiondown Para qu\'e usar mocking? }
	\bigskip
	\begin{itemize}
		\item Permite testear comportamiento
		\item Permite mejorar la cobertura
		\item Ayuda a aislar componentes eliminando dependencias
		\item Permite acelerar tests lentos (p. ej. al eliminar una DB)
		\item Permite prototipar un m\'odulo no implementado
	\end{itemize}
}

\only<2> {
	\textit{ \textquestiondown Google Mock? }
	\bigskip
	\begin{itemize}
		\item Mocking $\implies$ lenguaje de tipado din\'amico
		\item Hay pocos frameworks de mocking en C++
		\item Google Mock es una herramienta nueva
		\item No hace falta mantener los mocks a mano
		\item Es necesario entender DI, herencia y polimorfismo
	\end{itemize}
}

\bigskip
RTFM @ \url{http://code.google.com/p/googlemock}
\end{frame}

\begin{frame}{Un ejemplo y su test}
 Recordemos el c\'odigo original: \hyperlink{OriginalExample}{\beamerbutton{Click aqu\'i para verlo}}
\begin{itemize}
	\item Los singletons dificultan el testing
	\item La dependencia de la DB est\'a oculta
	\item Hace falta conocer la implementaci\'on
	\item Depencia en la implementaci\'on, no en la interfaz
\end{itemize}
\bigskip
\textit{Empezemos por algo f\'acil, refactorizemos la clase para que}
\begin{itemize}
	\item no dependa de la implementaci\'on de la BD
	\item podamos verificar las condiciones de error
\end{itemize}
\end{frame}


\begin{frame}[shrink=2,plain]{Un ejemplo, con 40\% mas DI}
\include{code/protocol2}
\end{frame}


\begin{frame}{Mocking Basics}
\textit{Eso fue f\'acil. Ahora, \textquestiondown c\'omo modificamos el test? \\
Recordemos que \ldots}
\begin{itemize}
	\item El objeto testeado depende de una interfaz
	\item El mock implementa la interfaz
	\item Al objeto testeado no le "importa" cual usa
	\item Es similar a tener mas de una implementaci\'on
\end{itemize}
\begin{center}
\pgfdeclareimage[height=3cm]{img/mocking/mocking_example}{img/mocking/mocking_example}
\pgfuseimage{img/mocking/mocking_example}
\end{center}
\end{frame}


\begin{frame}[shrink=2]{Un ejemplo, con 40\% mas DI y su test}
\include{code/protocol2_ut1}
\end{frame}


\begin{frame}[shrink=2]{Un ejemplo, con 40\% mas DI y su test: An\'alisis}
\textit{ Eso funciona pero }
\begin{itemize}
	\item No est\'a bueno tener que analizar los printfs
	\item No es automatizable
	\item Hay que mantener la implementaci\'on del mock
\end{itemize}
\bigskip
\textit{ Entonces nos conviene usar gMock }
\begin{itemize}
	\item MOCK\_METHOD\alert{N}(nombre, firma) $\implies$ declara un m\'etodo
	\item \alert{N} es el n\'umero de argumentos
	\item El m\'etodo debe ser virtual
\end{itemize}
\end{frame}


\begin{frame}[shrink=2]{Un ejemplo, con 40\% mas DI y su test}
\include{code/protocol2_ut2}
\end{frame}



\begin{frame}[t]{Definiendo comportamiento}
\textit{Compila pero no hace nada, nos falta definir las expectations}
\bigskip
\begin{itemize}
	\item EXPECT\_CALL es la forma mas b\'asica
	\item Se escribe EXPECT\_CALL(objeto \alert{,} m\'etodo)
	\item El m\'etodo es un "matcher"
	\item El "matcher" matchea los argumentos
	\item Luego de EXPECT\_CALL se define el comportamiento
	\item El comportamiento es lo que har\'a el mock al ser llamado
\end{itemize}
\only<1>{
	\begin{center}
	\pgfdeclareimage[height=3cm]{img/mocking/behaviour}{img/mocking/behaviour}
	\pgfuseimage{img/mocking/behaviour}
	\end{center}
}
\only<2>{
	\begin{center}
	EXPECT\_CALL( db, save\_error(\alert{MATCHER}) )
	\end{center}
}
\end{frame}



\begin{frame}[t]{Definiendo comportamiento: Matchers}
\begin{flushright}
	RTFM @ \href{http://code.google.com/p/googlemock/wiki/CheatSheet\#Matchers}{Google C++ Mocking Framework Cheat Sheet}
\end{flushright}
\begin{itemize}
	\item Un \alert{MATCHER} es una descripci\'on del argumento esperado
	\item Existen matchers para enteros, cadenas, etc. \\
		\only<2>{
			\begin{itemize}
			\item Para matchear un escalar podr\'iamos hacer \\
					\begin{center}
					EXPECT\_CALL( foo, bar(\alert{3}) )
					\end{center}
			\item Para matchear un rango podr\'iamos hacer \\
					\begin{center}
					EXPECT\_CALL( foo, bar(\alert{Gt(3)}) )
					\end{center}
			\item Para matchear una cadena podr\'iamos hacer \\
					\begin{center}
					EXPECT\_CALL( foo, bar(\alert{StrEq("foobar")}) )
					\end{center}
			\end{itemize}
		} \only<3>{
			\begin{itemize}
			\item Para matchear mas de una condici\'on \\
					\begin{center}
					EXPECT\_CALL( foo, bar(\alert{AllOf(Gt(3), Lt(10))}) )
					\end{center}
			\item Para matchear cualquier cosa \\
					\begin{center}
					EXPECT\_CALL( foo, bar(\alert{\_}) )
					\end{center}
			\end{itemize}
		}
	\item Misma funci\'on y 2 matchers $\implies$ expectations distintas
		\only<4>{
			\begin{itemize}
			\item Ac\'a hay dos expectations \\
					\begin{center}
					EXPECT\_CALL( foo, bar(\alert{2}) ) \\
					EXPECT\_CALL( foo, bar(\alert{3}) )
					\end{center}
					Para que est\'e OK hay que hacer \\
					\begin{center}
					foo.bar(2); foo.bar(3);
					\end{center}
			\item Aunque esto tambi\'en anda \ldots \\
					\begin{center}
					foo.bar(2); foo.bar(3); foo.bar(4);
					\end{center}
			\end{itemize}
		}
	\only<5> {
		\item Ya podemos escribir \\
			\begin{center}
			EXPECT\_CALL( db, save\_error(\alert{"recv"}) )
			\end{center}
		\item Todav\'ia nos falta definir las acciones!
	}
\end{itemize}
\only<1>{
	\begin{textblock*}{30mm}(-10mm,15mm)
	\pgfdeclareimage[height=3cm]{img/mocking/lipsum_mug}{img/mocking/lipsum_mug}
	\pgfuseimage{img/mocking/lipsum_mug}
	\end{textblock*}
}
\end{frame}


\begin{frame}[t]{Definiendo comportamiento: Acciones}
\begin{flushright}
	RTFM @ \href{http://code.google.com/p/googlemock/wiki/CheatSheet\#Cardinalities}{Google C++ Mocking Framework Cheat Sheet}
\end{flushright}
\begin{itemize}
	\item Un \alert{ACTION} define lo que har\'a un mock al ser llamado \\
		\only<2>{
			\begin{itemize}
			\item Por ejemplo: \\
				\begin{center} EXPECT\_CALL( foo, bar(\_) ).WillOnce(Return(42)); \end{center}
				Es un matcher que retornar\'a 42 la primer llamada. \\
				\textquestiondown Qu\'e har\'a en llamadas subsiguientes?
			\end{itemize}
		}
	\item Por cada matcher hay al menos una acci\'on \\
		\only<3>{
			\begin{itemize}
			\item Si no est\'a definida es una acci\'on por defecto, como \\
					 la segunda llamada del caso anterior.
			\item Para valores \alert{num\'ericos} es \alert{0}
			\item Para \alert{punteros} es \alert{NULL}
			\item Para \alert{objectos} dar\'a un \alert{error} y fallar\'a el test
			\item La acci\'on por defecto puede ser cambiada
			\end{itemize}
		}
	\item Por cada acci\'on hay al menos una cardinalidad \\
		\only<4>{
			\begin{itemize}
			\item Por defecto es \alert{1}
			\item La macro \alert{Times} define cardinalidad expl\'icita \\
				\begin{center} EXPECT\_CALL( foo, bar(\_) ).Times(42); \end{center}
			\item La cardinalidad queda implicita al definir m\'as de una acci\'on \\
				\begin{center} .WillOnce(Return(42)).WillOnce(Return(42)); \end{center}
			\item Podemos definir una acci\'on por defecto as\'i:
				\begin{center} .WillRepeatedly(Return(42)); \end{center}
			\end{itemize}
		}
	\only<5> {
		\item Ya podemos escribir \\
			\begin{center}
			EXPECT\_CALL( db, save\_error("recv") ) \\
				\alert{.WillOnce(} ACTION \alert{)};
			\end{center}
		\item Todav\'ia nos falta definir la acci\'on!
	}
\end{itemize}
\only<1>{
	\begin{textblock*}{30mm}(-10mm,15mm)
	\pgfdeclareimage[height=3cm]{img/mocking/action}{img/mocking/action}
	\pgfuseimage{img/mocking/action}
	\end{textblock*}
}
\end{frame}


\begin{frame}[t]{Definiendo comportamiento: Acciones II}
\begin{flushright}
	RTFM @ \href{http://code.google.com/p/googlemock/wiki/CheatSheet\#Actions}{Google C++ Mocking Framework Cheat Sheet}
\end{flushright}
\begin{itemize}
	\item Lo m\'as f\'acil es devolver un valor
	\only<2> {
		\begin{itemize}
		\item Tambi\'en se puede devolver "nada" \\
			\begin{center}
			EXPECT\_CALL( db, save\_error("recv") ) \\
				.WillOnce(\alert{Return()});
			\end{center}
		\item Las referencias son ezpecialez \\
			\begin{center}
			EXPECT\_CALL( foo, bar(\_) ) \\
				.WillOnce(\alert{ReturnRef(obj)});
			\end{center}
		\end{itemize}
	}
	\item Aunque no es dif\'icil hacer otra cosa
	\only<3> {
		\begin{itemize}
		\item Por ejemplo, asignar un valor
			\begin{center}.WillOnce(\alert{Assign(\&var, VALUE)})\end{center}
		\item O lanzar una excepci\'on
			\begin{center}.WillOnce(\alert{Throw(Exception)})\end{center}
		\end{itemize}
	}
	\item O llamar a alguien para que haga algo mas
	\only<4> {
		\begin{itemize}
		\item Por ejemplo, una funci\'on \footnote{Que debe tener la misma firma que el m\'etodo}
			\begin{center}.WillOnce(\alert{Invoke(f)})\end{center}
		\item O a un objeto
			\begin{center}.WillOnce(\alert{Invoke(object\_ptr, \&class::method})\end{center}
		\end{itemize}
	}
	\only<5> {
		\item Ahora si, estamos listos, nuestra expectation va a ser \\
			\begin{center}
			EXPECT\_CALL( db, save\_error("recv") ) \\
				.WillOnce(\alert{ Return() });
			\end{center}
	}
\end{itemize}
\only<1>{
	\begin{textblock*}{30mm}(-10mm,15mm)
	\pgfdeclareimage[height=3cm]{img/mocking/ostrich}{img/mocking/ostrich}
	\pgfuseimage{img/mocking/ostrich}
	\end{textblock*}
}
\end{frame}


\begin{frame}[shrink=2]{Un ejemplo, con 40\% mas DI y su test}
\include{code/protocol2_ut3}
\end{frame}

